Bei der Optimierung von Kommunikationssystemen geht es oft darum, konkrete Probleme auf abstrakte, mathematische 
Problemstellungen zurückzuführen, und man gelangt so in vielen Fällen zu Problemen in Graphen.
Eines dieser Probleme beschreibt \CallControl\ (auch Call-Admission-Control), bei dem
in einem Netzwerk eine Menge von Anfragen (sog.\ calls) gestellt werden und jede Anfrage gemäß einer bestimmten Zielsetzung
akzeptiert oder verworfen werden muss.
Es soll insbesondere sichergestellt werden, dass durch das Durchführen der akzeptierten Anfragen die Bandbreiten,
die das Netzwerk zur Verfügung stellt, nicht überstrapaziert werden.
Eine weitere Zielsetzung könnte dann beispielsweise sein, die Anzahl der akzeptierten Anfragen zu maximieren.

Da es meist schwierig ist, über willkürliche Graphen weitreichende Aussagen zu treffen, schränkt
man sich oft auf bestimmte Typen eines Graphen ein.
So beschränkt sich der Artikel \glqq Call Control in Rings\grqq\ \cite{paper} der vier Autoren Adamy, Ambühl, Anand und Erlebach auf Netzwerke,
die sich als Ring- oder Kettengraph darstellen lassen.

\subsection{Problemdefinition und Bezeichnungen}
Im Speziellen wird hier nur das sogenannte offline Szenario behandelt.
Das bedeutet, dass die Menge von Anfragen bereits zu Beginn feststeht und keine weiteren Anfragen während der
Laufzeit hinzukommen.

Zunächst führen wir einige wichtige Begriffe wie den eines (ungerichteten) Netzwerks ein:

\begin{definition}[Netzwerk]
	Sei $(V,E)$ ein ungerichteter Graph mit Knoten $V$ und Kanten $E$, und $c: E \to \set N$ eine Kapazitätsfunktion.
	Das Tupel $(V,E,c)$ heißt (ungerichtetes) Netzwerk und eine Kante $e$ besitzt darin die Kapazität $c(e)$.
\end{definition}

Die Anfragen werden nun als Pfade in einem Netzwerk betrachtet.
Dabei verbraucht ein Pfad eine Kapazitätseinheit einer Kante, falls er die Kante enthält.
Für die Zielsetzung, die Anzahl der akzeptierten Anfragen zu maximieren, nennen wir das entstehende Problem \CallControl.
Sind die Pfade gewichtet und wollen wir das Gesamtgewicht maximieren, reden wir von \WeightedCallControl.

\begin{definition}[\CallControl\ und \WeightedCallControl]
    Sei $(V,E,c)$ ein ungerichtetes Netzwerk und $P$ eine (Multi-)Menge von $m \in \set N$ Pfaden in $(V,E,c)$.
    Eine Menge $Q \subseteq P$ heißt {\em zulässig}, falls für jede Kante $e \in E$ die Anzahl aller Pfade in $Q$,
    die $e$ enthalten, höchstens $c(e)$ ist.
    
    {\em \CallControl} besteht darin, eine zulässige Menge $Q$ maximaler Mächtigkeit zu finden.
    Eine solche Menge nennen wir eine {\em optimale Lösung} des \CallControl-Problems.
    
    Gibt es außerdem eine Gewichtsfunktion $w \colon P \rightarrow \set R_+$, die jedem Pfad in $P$ ein 
    positives Gewicht zuweist, heißt das Problem, eine zulässige Menge $Q$ mit maximalem Gesamtgewicht
    $w(Q) \coloneqq \sum_{p \in Q} w(p)$ zu finden, {\em \WeightedCallControl} und $Q$ heißt 
    {\em optimale Lösung} des \WeightedCallControl-Problems.
\end{definition}

Eine {\em Kette} ist ein ungerichteter Graph $(V,E)$, der nur aus den Knoten und Kanten eines Pfades besteht, also einem
Weg mit den Kanten $E=\{(v_0, v_1),\dots,(v_{n-2},v_{n-1})\}$, sodass $v_i \neq v_j$ für $i \neq j$.
Dabei stellen wir uns vor, dass eine Kette entlang einer horizontalen Linie aufgezeichnet ist und die
Knoten bei $0$ beginnend von links nach rechts durchnummeriert sind. 
Entsprechend sagen wir, eine Kante bzw.\ ein Knoten liege {\em vor} einer anderen Kante bzw.\ einem anderen Knoten, 
falls diese(r) weiter links liegt.
Jeder Teilpfad $p$ kann in einer Kette durch zwei Endknoten identifiziert werden, einem Anfangsknoten $s_p$ und einem 
Zielknoten $t_p$, wobei der Anfangsknoten vor dem Zielknoten liegt.

Ein {\em Ring} ist ein ungerichteter Graph, der nur aus den Kanten und Knoten eines Kreises besteht, d.h.\ aus einem Weg 
mit den Kanten $E=\{(v_0, v_1),\dots,(v_{n-1},v_{n})\}$, sodass $v_0 = v_n$ und $v_i \neq v_j$ für alle anderen Paare $i,j$ 
mit $i \neq j$.
Betrachten wir einen Ring mit den Knoten $V=\{0, \dots , n-1\}$ aufgezeichnet auf einer Ebene, bei dem die Knoten nach
dem Uhrzeigersinn nummeriert wurden, identifizieren wir wieder jeden Pfad $p$ im Ring durch einen Anfangsknoten $s_p$ und einen Zielknoten 
$t_p$, wobei der Pfad alle Kanten vom Anfangsknoten aus im Uhrzeigersinn bis zum Zielknoten enthält.

\subsection{Motivation und Anwendungen}\label{subsec:motivationUndAnwendungen}

Oft trifft man in praktischen Anwendungen zwar auf die online-Variante des Problems, bei der jede Anfrage schon bei Eintreffen und
ohne Wissen über künftige Anfragen akzeptiert oder abgelehnt werden muss;
jedoch sind die Erkenntnisse über das offline-Problem auch hier relevant:
So ist es beispielsweise während der Konzeption eines Netzwerks hilfreich, die größtmögliche Belastbarkeit unter bestimmten 
Bedingungen untersuchen zu können.
Daneben gibt es auch Netzwerke, die für Anfragen Bandbreiten-Reservierungen im Voraus erlauben und diese dann in einem 
vorgeschriebenen Zeitintervall abarbeiten, was das offline-Problem in jedem dieser Intervalle abbildet.
Außerdem kann die optimale Lösung des offline-Problems auch als Vergleichswert bei der Bewertung von Lösungen des
online-Problems genutzt werden.

Eine direkte Anwendung bietet die zyklische Ablaufplanung.
Hier ist beispielsweise eine Menge an Aufgaben gegeben, die je in einem bestimmten Zeitintervall innerhalb eines Tages erledigt werden sollen.
Diese Zeitintervalle dürfen sich beliebig überschneiden oder sich sogar bis in den nächsten Tag erstrecken - sie müssen aber immer kürzer 
als 24 Stunden andauern.
Der Tag wird nun in so viele Intervalle $I_i$ aufgeteilt, sodass das Zeitintervall jeder Aufgabe als Vereinigung  
aufeinanderfolgender $I_i$ dargestellt werden kann.
Außerdem stehen jeden Tag im Zeitintervall $I_i$ genau $c_i$ Maschinen zur Verfügung und jede Maschine kann an jeder Aufgabe arbeiten, wobei zu jeder
Zeit an einer Aufgabe maximal eine Maschine arbeitet.
Vereinfachend wird angenommen, dass es egal ist, welche Maschine tatsächlich an einer Aufgabe arbeitet, und dass eine Übergabe einer laufenden
Aufgabe von Maschine zu Maschine, z.B.\ beim Wechsel eines Zeitintervalls, ohne Weiteres möglich ist.
Jedoch muss im vorgegebenen Zeitintervall zu jeder Zeit an einer Aufgabe gearbeitet werden, um sie zu erledigen.
Ziel ist es nun einen Ablaufplan zu finden, der die Anzahl an Aufgaben, die an einem Tag bearbeitet werden, maximiert.
Dieser Ablaufplan soll dann fortlaufend über mehrere Tage hinweg eingesetzt werden können.
Dazu kann \CallControl\ verwendet werden: Das Ringnetzwerk ist durch die Kanten $I_i$ mit den Kapazitäten $c_i$ der verfügbaren Maschinen
zu jedem Zeitintervall gegeben. Die dazugehörigen Aufgaben stellen die Anfragen im Netzwerk dar, die als Pfade auf dem Ring die Zeitintervalle
der Aufgaben abbilden.
Löst man nun diese Instanz des Call-Control-Problems, erhält man eine größtmögliche Menge an Pfaden, die keine Kapazitäten verletzt, und dadurch auch die
größtmögliche Menge an Aufgaben, die an einem Tag erledigt werden können.
