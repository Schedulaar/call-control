Bei der Optimierung von Kommunikationssystemen geht es oft darum, konkrete Probleme auf abstrakte, mathematische 
Problemstellungen zurückzuführen, und man gelangt so in vielen Fällen zu Problemen in Graphen.
Eines dieser Problem beschreibt das Call-Control-Problem (auch Call-Admission-Control-Problem), bei dem
in einem Netzwerk eine Menge von Anfragen (sog.\ calls) gestellt werden und jede Anfrage gemäß einer bestimmten Zielsetzung
akzeptiert oder verworfen werden muss.
Es soll insbesondere sichergestellt werden, dass durch das Durchführen der akzeptierten Anfragen die Bandbreiten,
die das Netzwerk zur Verfügung stellt, nicht überstrapaziert werden.
Eine weitere Zielsetzung könnte dann beispielsweise sein, die Anzahl der akzeptierten Anfragen zu maximieren.

Da es meist schwierig ist, über willkürliche Graphen weitreichende Aussagen zu treffen, schränkt
man sich oft auf bestimmte Typen eines Graphen ein.
So werden wir uns im Folgenden auf Netzwerke beschränken, die sich als Ring- oder Kettengraph darstellen lassen.

\subsection{Problemdefinition}
Im Speziellen behandeln wir nur das sogenannte Offline-Problem.
Das bedeutet, dass die Menge von Anfragen bereits zu Beginn feststeht und keine weiteren Anfragen während der
Laufzeit hinzukommen.

Zunächst wollen wir einige wichtige Begriffe wie dem eines (ungerichteten) Netzwerks einführen:

\begin{definition}[Netzwerk]
	Sei $(V,E)$ ein ungerichteter Graph mit Knoten $V$ und Kanten $E$, und $c: E \to \set N$ eine Kapazitätsfunktion.
	Das Tupel $(V,E,c)$ heißt (ungerichtetes) Netzwerk und eine Kante $e$ besitzt darin die Kapazität $c(e)$.
\end{definition}

Die Anfragen werden nun als Pfade in einem Netzwerk betrachtet.
Dabei verbraucht ein Pfad eine Kapazitätseinheit einer Kante, falls er die Kante enthält.
Für die Zielsetzung, die Anzahl der Anfragen zu maximieren, nennen wir das entstehende Problem \CallControl.
Sind die Pfade gewichtet und wir wollen das Gesamtgewicht maximieren, reden wir von \WeightedCallControl.

\begin{definition}[\CallControl\ und \WeightedCallControl]
    Sei $(V,E,c)$ ein ungerichtetes Netzwerk und $P$ eine (Multi-)Menge von $m \in \set N$ Pfaden in $(V,E,c)$.
    Eine Menge $Q \subseteq P$ heißt {\em geeignet}, falls für jede Kante $e \in E$ die Anzahl aller Pfade in $Q$,
    die $e$ enthalten, höchstens $c(e)$ ist.
    
    {\em \CallControl} besteht darin, eine geeignete Menge $Q$ maximaler Mächtigkeit zu finden.
    Eine solche Menge nennen wir dann eine {\em Lösung} des \CallControl-Problems.
    
    Gibt es außerdem eine Gewichtsfunktion $\omega \colon P \rightarrow \set R_+$, die jedem Pfad in $P$ ein 
    positives Gewicht zuweist, heißt das Problem, eine geeignete Menge $Q$ mit maximalem Gesamtgewicht
    $\omega(Q) \coloneqq \sum_{p \in Q} \omega(p)$ zu finden, {\em \WeightedCallControl}.
    Eine solche Menge $Q$ ist eine {\em Lösung} des \WeightedCallControl-Problems.
\end{definition}

Im Speziellen betrachten wir das Call-Control-Problem in Ringen und Ketten.
Eine {\em Kette} ist ein ungerichteter Graph $(V,E)$, der nur aus den Kanten und Knoten eines Weges besteht.
Ein {\em Ring} ist dabei ein ungerichteter Graph, der nur aus den Kanten und Knoten eines Zyklus' besteht.
Betrachten wir einen Ring $(V,E)$ mit $V=\{0, \dots , n-1\}$ aufgezeichnet auf einer Ebene, bei dem die Knoten nach
dem Uhrzeigersinn nummeriert werden, können wir jeden Pfad im Ring durch einen Anfangs- und Zielknoten identifizieren,
wobei der Pfad vom Anfangsknoten aus im Uhrzeigersinn bis zum Zielknoten \glqq verläuft\grqq.
Für einen Pfad $p$ in einem solchen Ring oder einer Kette, nennen wir $s_p$ den
Anfangsknoten und $t_p$ den Zielknoten des Pfades und schreiben $p=(s_p, t_p)$.


\subsection{Motivation und Anwendungen}\label{subsec:motivationUndAnwendungen}

Oft trifft man in praktischen Anwendungen zwar auf die online-Variante des Problems, bei der jede Anfrage schon bei Eintreffen und
ohne Wissen über künftige Anfragen akzeptiert oder abgelehnt werden muss.
Jedoch sind die Erkenntnisse über das offline-Problem auch hier relevant, um eine entsprechende Lösung für das
online-Problem entwerfen zu können.
Außerdem kann die optimale Lösung des offline-Problems als Vergleichspunkt bei der Bewertung von Lösungen des
online-Problems genutzt werden.

Eine Anwendung bietet das Problem der zyklischen Ablaufplanung:

\todo{Kette ist schon durchnummeriert definiert}