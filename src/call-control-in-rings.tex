Nachdem wir das Call-Control-Problem in Ketten optimal lösen können, betrachten wir nun das Problem in Ringen.
Sei also $(V, E)$ ein Ring mit $V=\{0,\dots,n-1 \}\}$ und $E=\{e_0, \dots, e_{n-1}\}$, wobei $e_i$ die
Knoten $i$ und $(i+1 \Mod n)$ verbindet.
Außerdem haben wir eine Menge $P=\{p_1, \dots, p_m\}$ an Pfaden gegeben, unter denen jeder Pfad $p$ einen
Anfangsknoten $s_p$ mit einem Zielknoten $t_p$ verbindet mit $s_p \neq t_p$.
Dabei enthält der Pfad alle Kanten, auf die man trifft, wenn man sich vom Anfangsknoten im Uhrzeigersinn - das heißt
bis auf $\mathrm{mod}~n$ mit aufsteigenden Knotennamen - zum Zielknoten bewegt.
Alle Knoten eines Pfades bis auf den Anfangs- und Zielknoten heißen {\em innere Knoten}.
Außerdem können wir ohne Einschränkung von $n \leq 2m$ ausgehen, da wir solche Knoten entfernen können, die nicht
Anfangs- oder Zielknoten eines Pfades sind.
Dabei ersetzen wir die zwei Kanten, die am zu entfernenden Knoten anliegen, durch eine einzige Kante, dessen
Kapazität das Minimum der vorherigen beiden ist.

\begin{figure}[htbp]
    \centering
    \def\svgwidth{250bp}
    \input{example_ring.pdf_tex}
    \caption{Ein Ringnetzwerk mit 8 Knoten (mod. nach \todo{[1]})}
\end{figure}

\todo{FIGURE} zeigt ein Beispiel eines Rings (innen) mit den gegebenen Pfaden (außen).
\todo{check if feasible set is defined}

Das Call-Control-Problem beschreibt hier wieder das Problem, eine möglichst große geeignete Teilmenge
$Q \subseteq P$ zu finden.
Dazu teilen wir die Menge der Pfade $P$ in zwei disjunkte Teilmengen $P_1$ und $P_2$.
Dabei bildet $P_1$ die Menge der Pfade in $P$, die den Knoten $0$ nicht als inneren Knoten haben, $P_2$
beinhaltet die restlichen Pfade, also $P \setminus P_1$.
Diese Partition können wir ganz einfach berechnen:
Sagen wir, dass jeder Pfad, dessen Zielknoten $0$ ist, eigentlich den Zielknoten $n$ hat, so können wir durch alle
Pfade $p$ iterieren und falls $s_p < t_p$ gilt, so ist $0$ kein innerer Knoten von $p$ und wird daher $P_1$
zugeordnet, sonst $P_2$.

Nun transformieren wir den Ring in eine Kette mit $2n$ Knoten:
Wir fertigen zwei Kopien der Kanten $e_1,\dotsc,e_n$ aus unserem Ring an und bilden eine Kette, indem wir die
beiden Kopien aneinanderhängen.
Die dazu gehörigen Knotennamen sind dabei $0, \dotsc, 2n-1$, wobei wir die Knoten $n, \dotsc, 2n-1$ durch
$0', \dotsc, (n-1)'$ als die zweite Kopie der Knoten kennzeichnen.
Die Pfade werden diesem Schema angepasst, sodass Pfade in $P_1$ nur Kanten der ersten Kopie belegen, wohingegen
jeder Pfad $p \in P_2$ aus zwei Teilpfaden besteht: Dem {\em Kopf}, der von $s_p$ bis nach $0'$ auf der ersten
Kopie der Kanten liegt, und dem {\em Schwanz}, der von $0'$ bis $t_p '$ auf der zweiten Kopie liegt.
In \todo{FIGURE} kann man den vorherigen Ring umgewandelt in die Kettenstruktur sehen.


\begin{figure}[htbp]
    \centering
    \def\svgwidth{240bp}
    \input{pic1.pdf_tex}
    \caption{svg image}
\end{figure}

Wir definieren im Weiteren für eine Menge $Q$ von Pfaden die Belastung $L_1(Q, e_i)$ bzw.\ $L_2(Q, e_i)$ der Kante
$e_i$ in der ersten bzw.\ zweiten Kopie als die Anzahl der Pfade von $Q$, die die erste bzw.\ zweite Kopie der Kante
$e_i$ beinhalten.
\begin{definition}[Profil]
    Sei $Q$ eine Menge von Pfaden.
    Die \todo{should i outsource this to a lemma?} monoton fallende Folge
    \[\pi_{Q}\colonequals  (L_2(Q, e_0), \dots, L_2(Q, e_{n-1}))\]
    der Belastungen der $n$ Kanten in den zweiten Kopien heißt Profil.
    Außerdem bezeichne $\pi_Q(e_i) \colonequals L_2(Q, e_i)$ und wir schreiben für zwei Profile $\pi$ und $\pi'$,
    dass $\pi \leq \pi'$, falls $\pi(e_i) \leq \pi'(e_i)$ für alle Kanten $e$.
\end{definition}
Wir bezeichnen eine Menge $Q$ von Pfaden als {\em kettengeeignet},
falls es in der oben konstruierten Kette keine Kapazität überschreitet, d.h.\ falls $L_1(Q, e) \leq c(e)$ und
$L_2(Q, e_i) \leq c(e)$ für alle Kanten $e$.
Gilt darüber hinaus $L_1(Q, e) + \pi(e) \leq c(e)$ für alle Kanten $e$, so heißt $Q$ {\em kettengeeignet zum Startprofil $\pi$}.
Anschaulich betrachtet, werden hier von den Kanten der ersten Kopie einige Kapazitäten bereits vom Startprofil belegt.
Betrachtet man jetzt nochmals die Konstruktion der Kette kann man leicht erkennen, dass eine Menge $Q$ von Pfaden geeignet
(im Ring) ist genau dann, wenn $Q$ kettengeeignet zum eigens generierten Startprofil $\pi_Q$ ist, welches ja gerade den
\todo{überstehenden} Pfaden entspricht.

\todo{check path sorting}

\subsection{Der Algorithmus}

Nachdem wir nun alle notwendigen Vorbereitung getroffen haben, können wir mit dem eigentlichen Algorithmus beginnen.
Dabei gehen wir wie folgt vor:
Wir suchen mit binärer Suche nach dem größten $k \in \{0, \dots, m\}$, sodass eine geeignete Teilmenge $Q_k \subseteq P$
mit $k$ Pfaden existiert, und geben schließlich $Q_k$ aus.

Die Prozedur für das Finden einer geeigneten $k$-elementigen Menge von Pfaden bauen wir dabei wie folgt auf.
Wir starten mit dem leeren Profil $\pi_0$, das überall 0 ist.
In jeder Runde $i \in \set N$ initialisieren wir die Kapazitäten der beiden Kopien jeder Kante $e$ mit $c(e)$, wobei wir
in den ersten Kopie von $c(e)$ noch $\pi_{i-1}(e)$ Kapazitäten abziehen.
Das Profil $\pi_{i-1}$ belegt also bereits einige Kanten in der ersten Kopie und eine Menge $Q$ ist mit diesen Kapazitäten
kettengeeignet genau dann, wenn sie auf den ursprünglichen Kapazitäten kettengeeignet zum Startprofil $\pi_{i-1}$ ist.
Nun bestimmen wir mit dem gierigen Algorithmus eine größtmögliche Menge $G$, die in der resultierende Kette
geeignet ist.
Enthält $G$ weniger als $k$ Pfade, so bricht die Prodzedur ab mit der Antwort, dass eine geeignete $k$-elementige Menge
nicht existiert.
Ansonsten sei $G_i$ die Menge der ersten $k$ Pfade von $G$ in gieriger Ordnung und $\pi_{i}$ das dadurch erzeugte
Profil.
Ist nun $\pi_i = \pi_{i-1}$, so meldet die Prozedur einen Erfolg mit $G_i$ als Rückgabe, sonst wird die nächste Runde
berechnet.

Ein Beispiel kann man in \todo{FIGURE} sehen.

\subsection{Korrektheit des Algorithmus'}\label{subsec:korrektheitCallControlInRings}
Um die Korrektheit des gesamten Algorithmus einzusehen, beweisen wir in den nächsten Lemmata, dass die Prozedur zum
Finden einer geeigneten $k$-elementigen Menge korrekt ist.
Dabei benutzen wir die vom Algorithmus generierten Mengen $G_i$, sowie dessen Profile $\pi_i$.

\begin{lemma}\label{lem:monotonousProfiles}
    Die Folge der Profile ist monoton wachsend, d.h.\ $\pi_i \leq \pi_{i+1}$ für alle $i \in \set N_0$.
\end{lemma}
\begin{proof}
    Durch Induktion: Für $i=0$ gilt die Aussage, da $\pi_0$ überall 0 ist.
    Angenommen, es gilt $\pi_{i-1} \leq \pi_i$.
    Nach Lemma~\ref{lem:optimalityGreedyAlgorithm} ist $G_{i}$ eine minimale Menge mit $k$ Pfaden, die auf der Kette mit den durch $\pi_{i}$
    reduzierten Kapazitäten geeignet ist.
    Da $G_{i-1}$ geeignet zum Startprofil $\pi_{i-1}$ ist und es gilt $\pi_{i-1} \leq \pi_i$, ist $G_{i-1}$ auch
    geeignet zum Startprofil $\pi_{i}$ und aufgrund der Minimalität von $G_i$ ist insbesondere $G_i \leq G_{i-1}$.
    Daraus folgt direkt $\pi_i \leq \pi_{i+1}$.
\end{proof}

Als Kondition dafür, dass unser Algorithmus hält, benötigen wir noch, dass die generierten Profile nicht
ewig wachsen können und, falls eine Lösung existiert, ab irgendeiner Runde konstant bleibt.

\begin{lemma}\label{lem:upperBoundProfiles}
    Falls eine im Ring geeignete Lösung $Q^*$ mit $k$ Pfaden existiert, dann ist $\pi_{Q^*}$ eine obere Grenze der
    generierten Profile $\pi_i$.
\end{lemma}
\begin{proof}
    Sei also $Q*$ eine geeignete Lösung mit $k$ Pfaden.
    Wir zeigen wieder per Induktion für $i \in \set N_0$, dass $\pi_i \leq \pi_{Q*}$.
    Für $i=0$ ist das offensichtlich, da $\pi_0$ überall 0 ist.
    Angenommen, $\pi_i \leq \pi_{Q*}$.
    Wie wir oben gesehen haben, ist $Q*$ kettengeeignet zum Startprofil $\pi_{Q^*}$, da es geeignet im Ring ist.
    Da $G_{i}$ mittels des gierigen Algorithmus' zum Startprofil $\pi_i$ ermittelt wird und $Q^*$ ebenfalls kettengeeignet
    zum Startprofil $\pi_i$ ist (da $\pi_i \leq \pi_{Q^*}$), gilt nach Lemma~\ref{lem:optimalityGreedyAlgorithm}, dass $G_{i} \leq Q^*$.
\end{proof}

Nun können wir die Korrektheit der Prozedur vervollständigen:

\begin{lemma}\label{lem:decisionProcedure}
    Die Prozedur liefert die korrekte Rückgabe in höchstens $n\cdot c(e_0)$ Runden.
\end{lemma}
\begin{proof}
    Für die Korrektheit unterscheiden wir, ob eine $k$-elementige geeignete Menge existiert.
    Nehmen wir an, es existiert keine solche Menge und der Algorithmus bricht nicht ab, sondern liefert eine Menge $Q_i$
    als Rückgabe.
    Dann müssen die letzten beiden Profile $\pi_{i-1}$ und $\pi_{i} = \pi_{Q_i}$ identisch gewesen sein.
    Da $Q_i$ aber durch den gierigen Algorithmus so berechnet wurde, dass es kettengeeignet zum Startprofil
    $\pi_{i-1}$ ist, ist es kettengeeignet zum eigenen Startprofil, insbesondere eine geeignete Menge im Ring, was im
    Widerspruch zur Annahme steht.

    Angenommen es existiere eine geeignete Menge $Q^*$ mit $k$ Elementen.
    Dann ist die Folge der generierten Profile $(\pi_i)_{i}$ nach Lemma~\ref{lem:upperBoundProfiles} mit $\pi_{Q^*}$
    nach oben beschränkt.
    Daher existiert zu jedem dieser Profile eine kettengeeignete $k$-elementige Menge (nämlich $Q^*$),
    weswegen der gierige Algorithmus nach Lemma~\ref{lem:optimalityGreedyAlgorithm} eine mindestens $k$-elementige Menge zurückgibt.
    Daher kann der Algorithmus nicht ohne Rückgabe abbrechen.
    Da die Folge $(\pi_i)_i$ nach Lemma~\ref{lem:monotonousProfiles} außerdem monoton wachsend ist, konvergiert sie nach
    maximal $\sum_{j=0}^{n-1}\pi_{Q^*}(e_j)$ Runden.
    Also existiert ein minimales $i\in \set N$ mit $\pi_i = \pi_{i-1}$ und der Algorithmus gibt $Q_i$ zurück.
    Da $\pi_{i-1} = \pi_{Q_{i}}$, ist $Q_i$ kettengeeignet zum eigenen Startprofil und damit eine geeignete Menge.
    Weil ein Profil eine monoton fallende Folge ist, bekommen wir zusätzlich die Abschätzung von maximal
    $\sum_{j=0}^{n-1}\pi_{Q^*}(e_j) \leq n \cdot \pi_{Q^*}(e_0) \leq n \cdot c(e_0)$ Runden.
\end{proof}
Unsere Ergebnisse können wir nun im folgenden Theorem zusammenfassen:
\begin{theorem}
    Das Call-Control-Problem in Ringen kann optimal in $O(m \cdot n \cdot c_{\min} \cdot \log m)$ Zeit gelöst werden,
    wobei $n$ die Anzahl der Knoten, $m$ die Anzahl der Pfade und $c_{\min}$ die kleinste Kantenkapazität ist.
\end{theorem}
\begin{proof}
    Jede Runde des Entscheidungsalgorithmus ruft einmal den gierigen Algorithmus auf.
    Dieser benötigt nach Theorem~\ref{lem:optimalityGreedyAlgorithm} eine Laufzeit von $O(n+m)$ und ,da o.E.\
    $n \leq 2m$, gilt $O(n+m) = O(m)$.
    Nach Lemma~\ref{lem:decisionProcedure} können wir in $O(n \cdot c(e_0))$ Runden eine geeignete $k$-elementige Menge finden.
    Dies kann auf $O(n \cdot c_{\min})$ reduziert werden, indem die Kanten und Knoten des Rings so benannt werden, dass
    die Kante $e_0$ derjenigen mit der geringsten Kapazität entspricht (dies gelingt in $O(m)$ Zeit).
    Mit binärer Suche benötigen wir schließlich noch $O(\log m)$ Entscheidungsrunden, um das größte $k \in \{0, \dots, m\}$
    mit einer geeigneten $k$-elementigen Menge zu finden.
\end{proof}













