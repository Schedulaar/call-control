%%
%% Author: markl
%% 2/26/2018
%%

% Preamble
\documentclass[11pt]{article}

% Packages
\usepackage{a4wide}
\usepackage[T1]{fontenc}
\usepackage[utf8]{inputenc}
\usepackage[german]{babel}
\usepackage{mathtools}
\usepackage{amsthm}
\usepackage{amssymb}

\title{\bfseries Call-Control in Ringnetzwerken}
\author{Michael Markl}
\date{Sommersemester 2018\\ an der\\ Universität Augsburg}

\theoremstyle{definition}
\newtheorem{definition}{Definition}[section]

% Document
\begin{document}
    \maketitle
    \clearpage

    \tableofcontents
    \clearpage

    \setcounter{section}{0}

    \section{Einführung}\label{sec:einführung}

    \begin{definition}[Call-Control-Problem]
        Sei $(V,E)$ ein ungerichteter Graph, dessen Kanten die Kapazitäten $c: E \to \mathbb{N}$ besitzen, und $P$ eine
        (Multi-)Menge von $m \in \mathbb{N}$ Pfaden in $(V,E)$. Eine Menge $Q \subseteq P$ heißt {\em möglich}, falls
        für jede Kante $e \in Q$ die Anzahl aller Pfade in $Q$, die $e$ enthalten, höchstens $c(e)$ ist.
        Die Pfade einer möglichen Menge heißen $akzeptiert$. Das {\em Call-Controll-Problem} besteht darin, die Anzahl
        der akzeptierten Pfade zu maximieren.
    \end{definition}

\end{document}
