%%
%% Author: markl
%% 2/26/2018
%%

% Preamble
\documentclass[11pt]{article}

% Packages
\usepackage{a4wide}
\usepackage[T1]{fontenc}
\usepackage[utf8]{inputenc}
\usepackage[german]{babel}
\usepackage{mathtools}
\usepackage{amsthm}
\usepackage{amssymb}
\usepackage{hyperref}
\usepackage[open]{bookmark}
\usepackage{amsmath}

\hypersetup{
colorlinks,
citecolor=black,
filecolor=black,
linkcolor=black,
urlcolor=black
}

\title{\bfseries Call-Control in Ringnetzwerken}
\author{Michael Markl}
\date{Sommersemester 2018\\ an der\\ Universität Augsburg}

\theoremstyle{definition}
\newtheorem{definition}{Definition}[section]

% Document
\begin{document}
    \maketitle
    \clearpage

    \tableofcontents

    \clearpage

    \setcounter{section}{0}

    \section{Einführung}\label{sec:einführung}

    \begin{definition}[Call-Control-Problem]
        Sei $(V,E)$ ein ungerichteter Graph, dessen Kanten die Kapazitäten $c: E \to \mathbb{N}$ besitzen, und $P$ eine
        (Multi-)Menge von $m \in \mathbb{N}$ Pfaden in $(V,E)$.
        Eine Menge $Q \subseteq P$ heißt {\em möglich}, falls für jede Kante $e \in E$ die Anzahl aller Pfade in $Q$,
        die $e$ enthalten, höchstens $c(e)$ ist.
        Die Pfade einer möglichen Menge heißen $akzeptiert$.
        Das {\em Call-Control-Problem} besteht darin, die Anzahl der akzeptierten Pfade zu maximieren.
    \end{definition}

    Im Speziellen betrachten wir das Call-Control-Problem in Ringen und Ketten.
    Ein {\em Ring} ist dabei ein ungerichteter Graph, der nur aus den Kanten und Knoten eines Zyklus' besteht.
    Eine {\em Kette} ist ein ungerichteter Graph, der nur aus den Kanten und Knoten eines Weges besteht.
    Betrachten wir einen Ring $(V,E)$ mit $V=\{0, \dots , n-1\}$ aufgezeichnet auf einer Ebene, bei dem die Knoten nach
    dem Uhrzeigersinn nummeriert werden, können wir jeden Pfad im Ring durch einen Anfangs- und Endknoten identifizieren,
    wobei der Pfad vom Anfangsknoten aus im Uhrzeigersinn bis
    zum Endknoten \glqq verläuft\grqq.

    \subsection{Motivation und Anwendungen}\label{subsec:motivationUndAnwendungen}
    Diese Problemstellung findet seine Anwendung darin, in einem (Kommunikations-)Netzwerk mit festen Bandbreiten jeder
    Kante und einer gegebenen Menge an Anfragen möglichst viele dieser Anfragen zuzulassen, wobei jede Anfrage auf allen
    Kanten ihres Pfades genau eine Bandbreiteneinheit aufbraucht und keine Kante ihre Bandbreite überschreiten kann.

    Oft trifft man zwar auf die online-Variante des Problems, das heißt, dass jede Anfrage schon bei Eintreffen und
    ohne Wissen über künftige Anfragen akzeptiert oder abgelehnt werden muss.
    Jedoch sind die Erkenntnisse über das offline-Problem auch hier relevant, um eine entsprechende Lösung für das
    online-Problem entwerfen zu können.
    Außerdem kann die optimale Lösung des offline-Problems als Vergleichspunkt bei der Bewertung von Lösungen des
    online-Problems genutzt werden.

    Eine Anwendung bietet das Problem der zyklischen Ablaufplanung:

    \section{Weiteres}\label{sec:weiteres}
    - WeightedCallProblem: CallProblem mit Gewichten 1


\end{document}
