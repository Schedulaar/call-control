%%
%% Author: markl
%% 2/26/2018
%%

% Preamble
\documentclass[11pt]{article}
% Packages
\usepackage[a4paper,margin=2.5cm]{geometry}
\usepackage[T1]{fontenc}
\usepackage{lmodern}
\usepackage[utf8]{inputenc}
\usepackage[german]{babel}
\usepackage{mathtools}
\usepackage{amsthm}
\usepackage{amssymb}
\usepackage{hyperref}
\usepackage[open]{bookmark}
\usepackage{amsmath}
\usepackage{colonequals}
\usepackage{xcolor}
\usepackage{titling}
\usepackage{graphicx}
\usepackage[skip=3pt]{caption}
\usepackage{subcaption}
\usepackage{lineno}

\renewcommand{\baselinestretch}{1.2}
\setlength{\belowcaptionskip}{-6pt}

\setlength{\droptitle}{-5em}
\setlength{\topsep}{6pt}
\setlength{\itemsep}{0pt}
\setlength{\partopsep}{0pt}

\setlength{\partopsep}{0pt}
\abovedisplayskip=0pt
\abovedisplayshortskip=0pt
\belowdisplayskip=0pt
\belowdisplayshortskip=0pt

\hypersetup{
colorlinks,
citecolor=black,
filecolor=black,
linkcolor=black,
urlcolor=black
}

\title{\bfseries Call-Control in Ringnetzwerken}
\author{Michael Markl}
\date{Seminar\\ \glqq Algorithmen und Datenstrukturen\grqq\\~ \\Sommersemester 2018\\ Universität Augsburg}

\theoremstyle{definition}
\newtheorem{definition}{Definition}[section]
\theoremstyle{theorem}
\newtheorem{lemma}{Lemma}[section]
\theoremstyle{theorem}
\newtheorem{theorem}{Theorem}[section]
\newcommand{\coloniff}{\vcentcolon\Longleftrightarrow}
\newcommand\todo[1]{\textcolor{red}{#1}}
\newcommand{\set}{\mathbb}
\newcommand{\find}{\text{\em find}}
\newcommand{\cost}{\text{\em cost}}
\newcommand{\capacity}{\text{\em cap}}
\newcommand{\Mod}{~\mathrm{mod}~}
\newcommand{\OPT}{\text{OPT}}
\newcommand{\CallControl}{Call-Control}
\newcommand{\WeightedCallControl}{Weighted-Call-Control}

\addto\captionsgerman{
\renewcommand{\figurename}{Abb.}
}

% Document
\begin{document}
    \maketitle

	\begin{linenumbers}
		\modulolinenumbers[5]
			
		\section{Einführung}\label{sec:einführung}
		Bei der Optimierung von Kommunikationssystemen geht es oft darum, konkrete Probleme auf abstrakte, mathematische 
Problemstellungen zurückzuführen, und man gelangt so in vielen Fällen zu Problemen in Graphen.
Eines dieser Probleme beschreibt \CallControl\ (auch Call-Admission-Control), bei dem
in einem Netzwerk eine Menge von Anfragen (sog.\ calls) gestellt werden und jede Anfrage gemäß einer bestimmten Zielsetzung
akzeptiert oder verworfen werden muss.
Es soll insbesondere sichergestellt werden, dass durch das Durchführen der akzeptierten Anfragen die Bandbreiten,
die das Netzwerk zur Verfügung stellt, nicht überstrapaziert werden.
Eine weitere Zielsetzung könnte dann beispielsweise sein, die Anzahl der akzeptierten Anfragen zu maximieren.

Da es meist schwierig ist, über willkürliche Graphen weitreichende Aussagen zu treffen, schränkt
man sich oft auf bestimmte Typen eines Graphen ein.
So beschränkt sich der Artikel \glqq Call Control in Rings\grqq\ \cite{paper} der vier Autoren Adamy, Ambühl, Anand und Erlebach auf Netzwerke,
die sich als Ring- oder Kettengraph darstellen lassen.

\subsection{Problemdefinition und Bezeichnungen}
Im Speziellen wird hier nur das sogenannte offline Szenario behandelt.
Das bedeutet, dass die Menge von Anfragen bereits zu Beginn feststeht und keine weiteren Anfragen während der
Laufzeit hinzukommen.

Zunächst führen wir einige wichtige Begriffe wie den eines (ungerichteten) Netzwerks ein:

\begin{definition}[Netzwerk]
	Sei $(V,E)$ ein ungerichteter Graph mit Knoten $V$ und Kanten $E$, und $c: E \to \set N$ eine Kapazitätsfunktion.
	Das Tupel $(V,E,c)$ heißt (ungerichtetes) Netzwerk und eine Kante $e$ besitzt darin die Kapazität $c(e)$.
\end{definition}

Die Anfragen werden nun als Pfade in einem Netzwerk betrachtet.
Dabei verbraucht ein Pfad eine Kapazitätseinheit einer Kante, falls er die Kante enthält.
Für die Zielsetzung, die Anzahl der Anfragen zu maximieren, nennen wir das entstehende Problem \CallControl.
Sind die Pfade gewichtet und wollen wir das Gesamtgewicht maximieren, reden wir von \WeightedCallControl.

\begin{definition}[\CallControl\ und \WeightedCallControl]
    Sei $(V,E,c)$ ein ungerichtetes Netzwerk und $P$ eine (Multi-)Menge von $m \in \set N$ Pfaden in $(V,E,c)$.
    Eine Menge $Q \subseteq P$ heißt {\em geeignet}, falls für jede Kante $e \in E$ die Anzahl aller Pfade in $Q$,
    die $e$ enthalten, höchstens $c(e)$ ist.
    
    {\em \CallControl} besteht darin, eine geeignete Menge $Q$ maximaler Mächtigkeit zu finden.
    Eine solche Menge nennen wir dann eine {\em Lösung} des \CallControl-Problems.
    
    Gibt es außerdem eine Gewichtsfunktion $\omega \colon P \rightarrow \set R_+$, die jedem Pfad in $P$ ein 
    positives Gewicht zuweist, heißt das Problem, eine geeignete Menge $Q$ mit maximalem Gesamtgewicht
    $\omega(Q) \coloneqq \sum_{p \in Q} \omega(p)$ zu finden, {\em \WeightedCallControl} und $Q$ heißt 
    {\em Lösung} des \WeightedCallControl-Problems.
\end{definition}

Eine {\em Kette} ist ein ungerichteter Graph $(V,E)$, der nur aus den Kanten und Knoten eines Pfades besteht, also einem
Weg mit den Kanten $E=\{(v_0, v_1),\dots,(v_{n-2},v_{n-1})\}$, sodass $v_i \neq v_j$ für $i \neq j$.
Dabei stellen wir uns vor, dass eine Kette entlang einer horizontalen Linie aufgezeichnet ist und die
Knoten bei $0$ beginnend von links nach rechts durchnummeriert sind. 
Entsprechend sagen wir, eine Kante bzw.\ ein Knoten liege {\em vor} einer anderen Kante bzw.\ einem anderen Knoten, 
falls diese(r) weiter links liegt.
Jeder Teilpfad $p$ kann in einer Kette durch zwei Endknoten identifiziert werden, einem Anfangsknoten $s_p$ und einem 
Zielknoten $t_p$, wobei der Anfangsknoten vor dem Zielknoten liegt.

Ein {\em Ring} ist ein ungerichteter Graph, der nur aus den Kanten und Knoten eines Kreises besteht, d.h.\ aus einem Weg 
mit den Kanten $E=\{(v_0, v_1),\dots,(v_{n-1},v_{n})\}$, sodass $v_0 = v_n$ und $v_i \neq v_j$ für alle anderen Paare $i,j$ 
mit $i \neq j$.
Betrachten wir einen Ring mit den Knoten $V=\{0, \dots , n-1\}$ aufgezeichnet auf einer Ebene, bei dem die Knoten nach
dem Uhrzeigersinn nummeriert wurden, identifizieren wir wieder jeden Pfad $p$ im Ring durch einen Anfangsknoten $s_p$ und einen Zielknoten 
$t_p$, wobei der Pfad alle Kanten vom Anfangsknoten aus im Uhrzeigersinn bis zum Zielknoten enthält.

\subsection{Motivation und Anwendungen}\label{subsec:motivationUndAnwendungen}

Oft trifft man in praktischen Anwendungen zwar auf die online-Variante des Problems, bei der jede Anfrage schon bei Eintreffen und
ohne Wissen über künftige Anfragen akzeptiert oder abgelehnt werden muss;
jedoch sind die Erkenntnisse über das offline-Problem auch hier relevant:
So ist es beispielsweise während der Konzeption eines Netzwerks hilfreich, die größtmögliche Belastbarkeit unter bestimmten 
Bedingungen untersuchen zu können.
Daneben gibt es auch Netzwerke, die für Anfragen Bandbreiten-Reservierungen im Voraus erlauben und diese dann in einem 
vorgeschriebenen Zeitintervall abarbeiten, was das offline-Problem in jedem dieser Intervalle abbildet.
Außerdem kann die optimale Lösung des offline-Problems auch als Vergleichswert bei der Bewertung von Lösungen des
online-Problems genutzt werden.

Eine direkte Anwendung bietet die zyklische Ablaufplanung.
Hier ist beispielsweise eine Menge an Aufgaben gegeben, die je in einem bestimmten Zeitintervall innerhalb eines Tages erledigt werden sollen.
Diese Zeitintervalle dürfen sich beliebig überschneiden oder sich sogar bis in den nächsten Tag erstrecken - sie müssen aber immer kürzer 
als 24 Stunden andauern.
Der Tag wird nun in so viele Intervalle $t_i$ aufgeteilt, sodass das Zeitintervall jeder Aufgabe als Vereinigung  
aufeinanderfolgender $t_i$ dargestellt werden kann.
Außerdem stehen jeden Tag im Zeitintervall $t_i$ genau $c_i$ Maschinen zur Verfügung und jede Maschine kann jede Aufgabe erledigen, wobei immer nur 
eine Maschine an einer Aufgabe gleichzeitig arbeiten kann.
Vereinfachend wird angenommen, dass es egal ist, welche Maschine tatsächlich an einer Aufgabe arbeitet, und dass eine Übergabe einer laufenden
Aufgabe von Maschine zu Maschine, z.B.\ beim Wechsel eines Zeitintervalls, ohne Weiteres möglich ist.
Ziel ist es nun einen Ablaufplan zu finden, der die Anzahl an Aufgaben, die an einem Tag bearbeitet werden, maximiert.
Dieser Ablaufplan soll dann fortlaufend über mehrere Tage hinweg eingesetzt werden können.
Dazu kann \CallControl\ verwendet werden: Das Ringnetzwerk ist dabei durch die Kanten $t_i$ mit den Kapazitäten $c_i$ der verfügbaren Maschinen
zu jedem Zeitintervall gegeben. Die dazugehörigen Aufgaben stellen dabei die Anfragen im Netzwerk dar, die als Pfade auf dem Ring die Zeitintervalle
der Aufgaben abbilden.
Löst man nun das Call-Control-Problem, erhält man eine größtmögliche Menge an Pfaden, die keine Kapazitäten verletzt, und dadurch auch die
größtmögliche Menge an Aufgaben, die an einem Tag bearbeitet werden können.

			
		\section{Optimale Lösung des \CallControl-Problems in Ketten}\label{sec:call-control-in-chains}
		Dieser Abschnitt analysiert den einfacheren Fall des Problems, nämlich den in einer Kette:
Sei $(V,E)$ eine Kette mit Kapazitäten $c: E \to \set N$ und $V=\{0,\dots,n\}$, sodass für $i \in \{0,\dots,n-1\}$
die Kante $e_i$ die Knoten $i$ und $i+1$ verbindet.
Außerdem sei eine Menge $P$ von Pfaden in $(V,E)$ gegeben.
\begin{definition}[Gierige Ordnung]
    Auf einer Menge $P$ von Pfaden in einer Kette nennen wir eine Totalordnung $\leq_G$ sowie die zugehörige strenge Totalordnung
    Ordnung $<_G$ {\em gierig},
    falls sie die Pfade nach ihren Zielknoten aufsteigend ordnet, das heißt, falls
    $\forall p, q \in P \colon t_p < t_q \Rightarrow p <_G q$.
\end{definition}
Im Folgenden bezeichne $\leq_G$ eine feste gierige Ordnung auf $P$.
Die Berechnung der optimalen Lösung für \CallControl\ in der Kette erfolgt mit dem {\em gierigen Verfahren}:
Dieses nimmt an, dass die Pfade $P$ bereits in gieriger Ordnung $\leq_G$ sortiert sind,
und verwaltet eine Menge der von ihm akzeptierten Pfade, die zu Beginn leer ist.
Es bearbeitet alle Pfade in der gegebenen Sortierung und fügt einen Pfad zu den akzeptierten Pfaden hinzu, falls die
akzeptierten Pfade dadurch keine der Kantenkapazitäten verletzen, das heißt, falls die akzeptierten Pfade eine zulässige Menge bleiben.
Ansonsten wird der Pfad abgelehnt.
Nachdem alle Pfade bearbeitet wurden, endet das Verfahren mit den akzeptierten Pfaden als Rückgabe.

Jede Menge $A \subseteq P$ von $k$ Pfaden kann als eine Menge $\{a_1, \dots, a_k\}$ dargestellt werden mit
$a_1 <_G \dots <_G a_k$.
Für solche $A=\{a_1,\dots,a_k\}$ und $B=\{b_1,\dots,b_k\}$ schreibe man dann $A \leq_G B$, falls $\forall i \in \{1,\dots,k\}\colon a_i \leq_G b_i$.
Dabei nennen wir eine zulässige Menge $A$ {\em minimal}, falls $A \leq_G B$ für jede gleich-mächtige, zulässige Menge $B$.
\begin{lemma}[Optimalität des gierigen Verfahrens]\label{lem:optimalityGreedyAlgorithm}
    Existiert bei einer Menge $P$ von Pfaden einer Kette eine zulässige Teilmenge mit $k \in \set N$ Pfaden, so ist die
    Menge der in gieriger Ordnung $\leq_G$ kleinsten $k$ Pfade, die das gierige Verfahren berechnet, eine minimale Menge.
\end{lemma}
\begin{proof}
    Sei $Q_0$ eine zulässige Menge mit $k$ Pfaden.
    Wir nennen $G$ die Menge der in $\leq_G$ kleinsten $k$ Pfade, die das gierige Verfahren berechnet, und
    transformieren $Q_0$ schrittweise zu $Q_k = G$.
    Dabei soll für $i \in \{1,\dots\,k\}$ gelten, dass $Q_i$ zulässig ist sowie $Q_i \leq_G Q_{i-1}$.
    Dadurch ist auch $G$ zulässig und durch die Transitivität von $\leq_G$ auf Mengen von Pfaden folgt $G \leq_G Q_0$,
    also insbesondere die Minimalität von $G$.

    Als weitere Invariante nehmen wir auf, dass die ersten $i$ Pfade (in gieriger Ordnung) von $Q_i$ mit denen von $G$
    übereinstimmen.
    Für $i=0$ gelten alle Voraussetzungen.
    Nehmen wir also an, $Q_{i-1}$ sei zulässig und stimme auf den ersten $i-1$ Pfaden mit $G$ überein.
    Den $i$-ten Pfad von $G$ nennen wir $p=(s_p, t_p)$.

    Für den Fall $p \in Q_{i-1}$ ist $p$ auch bereits der $i$-te Pfad von $Q_{i-1}$, da das Verfahren nach
    gieriger Ordnung vorgeht und $Q_{i-1}$ mit $G$ auf den ersten $i-1$ Pfaden übereinstimmt.
    Setzt man $Q_i \coloneqq Q_{i-1}$, bleiben alle Invarianten erhalten.

    Sonst gilt $p \notin Q_{i-1}$ und die Menge der Pfade $q \in Q_{i-1}$ mit $q >_G p$ ist nicht leer.
    Von diesen Pfaden sei $q$ ein solcher mit kleinstem Startknoten $s_q$, der in $Q_{i-1}$ an $j$-ter Stelle steht
    (insb.\ $j \geq i$).
    Setzt man nun $Q_{i} \coloneqq (Q_{i-1} \cup \{ p \}) \setminus \{ q \} $, so stimmen $Q_{i}$ und $G$ an den
    ersten $i$ Stellen wieder überein und alle Pfade, die in $Q_{i-1}$ an den Stellen $i$ bis $j-1$ stehen, rutschen
    in $Q_i$ eine Stelle weiter an die Positionen $i+1$ bis $j$.
    Insbesondere gilt also $Q_i \leq_G Q_{i-1}$.
    Es bleibt zu zeigen, dass $Q_i$ wieder zulässig ist, wozu die Kapazitäten der einzelnen Kanten betrachtet wird:
    Die Kanten, die links der beiden Anfangsknoten $s_q$ und $s_p$ stehen, sind nicht betroffen, da aufgrund der Minimalität
    von $s_q$ dort nach $q$ keine weiteren Kapazitäten benötigt werden.
    Ist $s_q$ kleiner als $s_p$, so sparen die Kanten zwischen $s_q$ und $s_p$ sogar eine Kapazität ein.
    Ist andersrum $s_p$ kleiner als $s_q$, verletzt $Q_i$ auf den Zwischenkanten trotzdem keine Kapazitäten, da
    hier nach $p$ aufgrund der Minimalität von $s_q$ keine weiteren Pfade eine Kapazität verbrauchen und $p$ vom
    Verfahren akzeptiert wurde, weil bis zum Pfad $p$ keine Kapazität verletzt wurde.
    Die Kanten, bei denen sich $p$ und $q$ überschneiden, spielen offensichtlich keine Rolle und die Kanten
    vom Zielknoten von $p$ bis zum Zielknoten von $q$ benötigen wieder eine Kapazität weniger.
\end{proof}

Daraus folgt, dass das gierige Verfahren das Call-Control-Problem in Ketten optimal löst:
Sei $Q$ eine optimale Lösung, also eine möglichst große, zulässig Teilmenge von Pfaden in $P$, so liefert das gierige
Verfahren nach Lemma~\ref{lem:optimalityGreedyAlgorithm} eine gleichmächtige, sogar minimale Lösung.

Eine einfache Implementierung dieses Verfahrens könnte in $\mathcal O(m \cdot n)$ Zeit erfolgen, wobei $m$ die Anzahl der gegebenen
Pfade und $n$ die Anzahl der Kanten ist, indem einfach für jeden Pfad einzeln an jeder Kante überprüft wird, ob durch
die Hinzunahme die Kapazität überschritten wird.
Jedoch lässt sich dieser Zeitaufwand sogar auf $\mathcal O(n+m)$ verringern, wie wir in den nächsten Abschnitten erkennen werden.
Dabei werden wir zunächst einen Algorithmus kennenlernen, bei dem vorausgesetzt wird, dass alle Kanten die gleiche
Kapazität haben, und diesen danach an unser Ausgangsproblem anpassen.

\subsection{Gieriger Algorithmus für gleiche Kapazitäten von Carlisle und Lloyd}\label{subsec:algorithmusGleicheKapazitäten}
Sei wieder $(V,E)$ eine Kette und $P$ eine Menge von $m$ Pfaden in $(V,E)$, wobei alle Kanten in $E$ nun die
feste Kapazität $C \in \set N$ besitzen.
Dabei können wir ohne Beschränkung der Allgemeinheit von $C \leq m$ ausgehen (sonst könnten wir einfach alle Pfade ohne Gefahr
akzeptieren).
Der Algorithmus, den Carlisle und Lloyd in~\cite{carlisle} beschreiben, hat eine etwas abgewandelte Problemformulierung:
Es wird statt von Pfaden in einer Kette von Intervallen auf der Zahlengeraden gesprochen.
Das Verfahren soll nun möglichst viele dieser Intervalle in $C$ verschiedene Farben färben, wobei sich zwei Intervalle derselben
Farbe nie überschneiden dürfen.
Wie man schnell erkennt, entspricht die größte Menge der gefärbten Intervalle im Ursprungsproblem gerade der größten Teilmenge der Pfade,
in der sich maximal $C$ Pfade an einer Kante überschneiden dürfen, bei der also jede Kante die Kapazität $C$ hat.
Statt also die Intervalle zu färben, reden wir im Folgenden davon, die Pfade entsprechend zu färben.

Der Algorithmus führt zunächst $C$ weitere sogenannte {\em virtuelle Pfade} ein, die in gieriger Ordnung
vor den eigentlichen Pfaden $P$ stehen und jeweils in einer der $C$ Farben gefärbt sind.
Dann werden alle Pfade in gieriger Ordnung verarbeitet.
Dabei wird zu jeder Farbe $c$ der aktuelle {\em Anführer} gespeichert, das heißt, der in gieriger Ordnung größte, bereits
verarbeitete Pfad mit der Farbe $c$.
Die virtuellen Pfade stellen dabei die initialen Anführer der Farben dar.
Der Algorithmus sucht jetzt für jeden Pfad $p$ den größten Anführer, der sich nicht mit $p$ überschneidet.
Ein solcher Pfad heißt {\em optimaler Anführer von $p$}.
Findet er einen solchen, wird $p$ akzeptiert, in die Farbe seines optimalen Anführers gefärbt und damit neuer Anführer
der Farbe.
Findet er keinen, gibt es keine freie Farbe, das heißt das Akzeptieren des Pfades würde eine Kapazität verletzen und der
Pfad wird abgelehnt.
Eine solche Färbung mit $C = 2$, kann man in Abbildung~\ref{fig:k-coloring} sehen.

\begin{figure}[htbp]
	\centering
	\def\svgwidth{250bp}
	\input{k-coloring.pdf_tex}
	\caption{Eine Färbung mit zwei Farben; weiße Pfade sind ungefärbt. (mod. nach~\cite{paper})}
	\label{fig:k-coloring}
\end{figure}

Speichert man allerdings zu jeder Farbe einfach den jeweiligen Anführer direkt ab, würde das Berechnen eines optimalen
Anführers für jeden Pfad mindestens $\mathcal O(\log C)$ Zeit benötigen und damit würden wir das Ziel einer gesamt-linearen Laufzeit
$\mathcal O(m)$ verpassen.

Stattdessen verwenden Carlisle und Lloyd in ihrem Algorithmus eine spezielle Union-Find-Struktur, die sogenannte
Static-Tree-Set-Union-Struktur aus~\cite{static-tree-set-union}, bei der die möglichen Vereinigungen bereits zu Beginn feststehen und in einem Baum
dargestellt werden können, wodurch $m$ union- und find-Aufrufe bei $n$ Elementen in einer Laufzeit von $\mathcal O(m + n)$ erfolgen können.
Außerdem geht man beim Algorithmus davon aus, dass eine bereits aufsteigend sortierte Liste der Endknoten aller Pfade existiert.
Diese Liste hat also $2\cdot m$ Einträge, d.h.\ ein Punkt kann für verschiedene Pfade öfters vorkommen, dabei sind doppelte Einträge entsprechend der gierigen
Ordnung der zugehörigen Pfade sortiert.

Genauer fügt der Algorithmus in gieriger Ordnung an erster Stelle (noch vor den virtuellen Pfaden) einen weiteren {\em fiktiven Anführer $f$}
ein, der ermöglicht, nicht-akzeptierte Pfade in der Datenstruktur zu behandeln.
Zu Beginn ist dann jeder Pfad in einer eigenen Gruppe der Union-Find-Instanz.
Wir erhalten folgende Invarianten:
Der Repräsentant jeder Gruppe ist immer der kleinste Pfad (in gieriger Ordnung) innerhalb der Gruppe.
Ist ein Pfad noch nicht verarbeitet, so ist er in einer Einzelgruppe - die virtuellen Pfade und der fiktive Anführer
zählen jedoch bereits als verarbeitet.
Jede andere, sogenannte {\em aktive Gruppe} enthält nur (in gieriger Ordnung) aufeinanderfolgende Pfade.
Der Repräsentant einer aktiven Gruppe ist dabei entweder ein Anführer einer Farbe oder der fiktive Anführer.
Dementsprechend gibt es zu jeder Zeit genau $C+1$ aktive Gruppen, wobei die Gruppe des fiktiven
Anführers immer die Gruppe mit den in gieriger Ordnung kleinsten Pfaden bleibt.

Zunächst berechnet der Algorithmus zu jedem Pfad $p \in P$ den {\em bevorzugten Anführer von $p$}, also den größten Pfad
in gieriger Ordnung, dessen Zielknoten nicht nach dem Anfangsknoten $s_p$ von $p$ steht.
Das kann man mit einfachem Durchlaufen der Liste aller Endknoten in $P$ erreichen, also in $\mathcal O(n)$ Zeit.
Es ist offensichtlich, dass kein optimaler Anführer von $p$ in gieriger Ordnung größer als der bevorzugte Anführer von
$p$ sein kann.

In Abbildung~\ref{fig:union-find-setup} sieht man die entstehende Ausgangssituation für ein Beispiel mit 6 Pfaden.

\begin{figure}[htbp]
	\centering
	\def\svgwidth{250bp}
	\input{union-find-setup.pdf_tex}
	\caption{Die Ausgangssituation: Mit den Pfeilen wird für jeden Pfad der bevorzugte Anführer markiert; $v_1$ und $v_2$ sind die virtuellen Pfade und $f$ ist der fiktive Anführer.}
	\label{fig:union-find-setup}
\end{figure}

Der Algorithmus bearbeitet danach jeden Pfad $p \in P$ nach gieriger Reihenfolge, wobei er jeweils wie folgt vorgeht.
Es wird der bevorzugte Anführer $q$ von $p$ betrachtet:
Ist $\find(q)$, also der Repräsentant der Gruppe von $q$, der fiktive Anführer $f$, so sind alle Pfade in $P$, die kleinergleich dem bevorzugten Anführer $q$
sind und damit Kandidaten für einen optimalen Anführer von $p$ wären, auch in der Gruppe von $f$. Daher kann es keinen optimalen Anführer zu $p$ geben,
da alle Anführer die Repräsentanten der anderen $C$ aktiven Gruppen sind und daher in gieriger Ordnung nach dem bevorzugten Anführer $q$ kommen.
Daher wird $p$ abgelehnt und die Gruppe von $p$ mit der Gruppe des Vorgängers von $p$ in gieriger Ordnung vereinigt.
Die Invarianten bleiben erhalten, da wieder $C+1$ Gruppen unter den verarbeiteten Pfaden existieren mit den gleichen
Anführern.

Ist $\find(q)$ jedoch ein Anführer einer Farbe, so ist es mit Sicherheit der optimale Anführer von $p$, da es der größte
Anführer ist, der noch kleinergleich dem bevorzugten Anführer von $p$ ist (da die Pfade in den Gruppen
aufeinanderfolgend sind).
Also wird $p$ akzeptiert, in dieser Farbe gefärbt und damit neuer Anführer der Farbe.
Da $\find(q)$ nun kein Anführer mehr ist, wird seine Gruppe aufgelöst und mit der Gruppe des Vorgängers von $\find(q)$
vereinigt.
Wieder erhalten wir alle Invarianten.

\begin{figure}[htbp]
	\centering
	\def\svgwidth{250bp}
	\input{union-find-structure.pdf_tex}
	\caption{Der Verlauf der Union-Find-Instanz für das Beispiel aus Abb.~\ref{fig:union-find-setup}: Jedes Rechteck stellt eine Gruppe der Instanz dar. Der Repräsentant einer Gruppe steht am linken Rand. Anführer sind durch Sterne gekennzeichnet, Färbungen durch Punkte bzw. Sterne in schwarz, grau oder weiß (für nicht akzeptiert).}
	\label{fig:union-find-structure}
\end{figure}

In Abbildung~\ref{fig:union-find-structure} sieht man den Verlauf der Union-Find-Instanz für das Beispiel aus Abbildung~\ref{fig:union-find-setup}.
Insbesondere wird erkennbar, dass wir nur Vereinigungen entlang einer Kette machen, was uns ermöglicht, die effizientere Static-Tree-Set-Union-Struktur
aus~\cite{static-tree-set-union} zu verwenden.
Eine genauere Analyse dieses Verfahrens und dessen Laufzeit $\mathcal O(m)$ beschreiben Carlisle und Llyod in~\cite{carlisle}.

\subsection{Gieriger Algorithmus mit willkürlichen Kapazitäten}\label{subsec:anpassenAnWillkürlicheKapazitäten}

Nachdem wir einen effizienten Algorithmus für identische Kapazitäten gefunden haben, wollen wir nun das eigentliche Problem mit
willkürlichen Kapazitäten $c: E \mapsto \set N$ durch geschicktes Anpassen des Algorithmus aus Abschnitt~\ref{subsec:algorithmusGleicheKapazitäten}    lösen.
Sei $C \coloneqq \max_{e \in E} c(e)$ die maximale Kapazität aller Kanten.
Die Idee ist es, den Algorithmus für Kanten mit identischer Kapazität $C$ anzuwenden und dabei für jede Kante $e$ die $C - c(e)$
hinzugewonnen Kapazitäten mit Platzhalterpfaden zu befüllen.
Werden alle Platzhalterpfade vom Algorithmus akzeptiert, so können wir sicher sein, dass wir nach Entfernen der
Platzhalterpfade eine optimale Lösung für willkürliche Kapazitäten gefunden haben.
Wir müssen also insbesondere dafür sorgen, dass all unsere Platzhalter eingefärbt werden.
Eine wie im vorigen Abschnitt aufsteigend sortierte Liste der Endknoten aller Pfade berechnen wir nun allerdings selbst mit einem
Aufwand von $\mathcal O(n+m)$, z.B.\ durch Sortieren durch Zählen.
Dabei speichern wir zu jedem Endknotens in der Liste eine Referenz auf seinen zugehörigen Pfad ab.
Insbesondere können wir dann für jeden Eintrag in der Liste entscheiden, ob der Knoten ein Anfangs- oder ein Zielknoten seines zugehörigen Pfades ist.

\begin{figure}[htbp]
	\centering
	\def\svgwidth{250bp}
	\input{dummy_paths.pdf_tex}
	\caption{Platzhalterpfade in schwarz dargestellt (mod. nach~\cite{paper})}
	\label{fig:dummy-paths}
\end{figure}

Wir beschäftigen uns zunächst näher mit dem Hinzufügen der Platzhalter:
Dazu wird die Liste aller Knoten der Kette durchlaufen und gleichzeitig berechnet, wie viele Platzhalterpfade an jedem Knoten
beginnen bzw.\ enden müssen (siehe Abbildung~\ref{fig:dummy-paths}):
Beim ersten Knoten $v_0$ müssen $C - c(e_1)$ Platzhalterpfade beginnen und wir schreiben $(C - c(e_1))$ mal $v_0$ in einen
leeren Keller $K$.
Ist $c(e_i) > c(e_{i+1})$, beginnen beim Knoten $v_{i}$ weitere Platzhalter und $v_i$ wird genau $(c(e_i) - c(e_{i+1}))$ mal in $K$ gelegt.
Ist $c(e_i) < c(e_{i+1})$, so müssen Platzhalterpfade bei $v_i$ enden und es werden $c(e_{i+1}) - c(e_i)$ Elemente aus K geholt, die jeweils als Anfangsknoten mit $v_i$
als Zielknoten einen der Platzhalterpfade bilden.
Beim letzten Knoten $v_{n-1}$ bilden dann alle verbleibenden Knoten aus $K$ als Anfangsknoten mit $v_{n-1}$ als Zielknoten die restlichen Platzhalter.

Da wir allerdings insgesamt eine Laufzeit von $\mathcal O(n+m)$ zum Ziel haben, können wir das beschriebene Verfahren
nicht einfach übernehmen:
Bei starken Schwankungen der einzelnen Kapazitäten kann es hier passieren, dass bis zu $\Omega(m\cdot n)$
Platzhalter hinzugefügt werden müssen, was unsere Ziellaufzeit übersteigt (siehe Abbildung~\ref{fig:dummy-paths-too-many}).
Jedoch wird schnell ersichtlich, dass in solchen Fällen viele Platzhalter unnötigerweise platziert werden:
Hat beispielsweise die Nachfolgekante eines Knoten $v$ eine sehr hohe Kapazität, wobei die Vorgängerkante nur eine sehr
niedrige Kapazität besitzt, so müssten sehr viele Pfade den Anfangsknoten $v$ haben, um die Kapazität der
Nachfolgekante auszunutzen.
Da wir die Pfade bereits kennen, können wir also in vielen Fällen die Kapazität von solchen Nachfolgekanten verringern und
damit Schwankungen in den Kapazitäten etwas ausgleichen.

\begin{figure}[htbp]
	\centering
	\def\svgwidth{250bp}
	\input{dummy_paths_too_many.pdf_tex}
	\caption{Bei jeder zweiten Kante werden $C-1=\Omega(m)$ Platzhalter eingefügt (mod. nach~\cite{paper})}
	\label{fig:dummy-paths-too-many}
\end{figure}

Um diese Erkenntnis umzusetzen, definieren wir eine neue Kapazitätsfunktion $c'$ mit
\[
	c'(e_i) =
	\begin{cases}
		\min(c(e_0), n_0) &\quad\text{für}\ i=0\\
		\min(c(e_i), c'(e_{i-1}) + n_i) &\quad\text{für}\ i \geq 1
	\end{cases}
\]
wobei $n_i$ die Anzahl der Pfade in P mit Anfangsknoten $v_i$ ist.
Diese können wir in $\mathcal O(n+m)$ Zeit berechnen: Dazu gehen wir einmal durch die sortierte Liste der $2m$ Endknoten
aller Pfade und zählen dabei für jeden Knoten $v_i$ der Kette alle Anfangsknoten $v$ in der Liste (überspringe Zielknoten) mit $v = v_i$.
Dies liefert $n_i$, womit wir auch $c'(e_i)$ berechnen können.
Durch das Verwenden von $c'$ statt $c$ wird das Problem auch nicht verändert, da nur Kapazitäten entfernt
werden, die die Pfade $P$ ohnehin nicht nutzen könnten.
Insbesondere ist also jede zulässige Menge auf den Kapazitäten $c$ auch eine zulässige Menge auf den Kapazitäten $c'$.
Das nächste Lemma soll nun aufklären, ob unser Vorgehen die Anzahl an Platzhaltern ausreichend verringern konnte:

\begin{lemma}
    Mit den neuen Kapazitäten $c'$ werden nur noch $\mathcal O(m)$ Platzhalterpfade erzeugt.
\end{lemma}
\begin{proof}
    Wir betrachten den Gesamtzuwachs der Kapazitäten $I = \sum_{i = 0}^{n-2} I_i$ als Summe aller Zuwächse $I_i$ am
    Knoten $v_i$ mit $I_0 = c'(e_0)$ und $I_i = \max\{c'(e_i) - c'(e_{i-1}), 0\}$ für $i \geq 1$.
    Ein Anstieg von $c'$ in $e_i$ um $I_i \in \set N_0$ bedeutet, dass $n_i \geq I_i$, da per Definition
    $c'(e_i) \leq c'(e_{i-1}) + n_i$.
    Insbesondere ist also $I$ beschränkt durch die Anzahl an Pfaden $m$.
    Ein Platzhalterpfad wird außerdem nur dann erzeugt, wenn der Algorithmus auf einen Zuwachs in den Kapazitäten (höchstens $I$)
    oder auf das Ende der Kette (höchstens $C$) stößt.
    Das heißt es werden höchstens $I + C = \mathcal O(m)$ Platzhalterpfade erzeugt.
\end{proof}

Insbesondere gelingt also die Anpassung der Kapazitäten sowie das Auffüllen mit Platzhalterpfaden in $\mathcal O(n+m)$ Zeit.
Nun müssen wir uns nur noch der Herausforderung stellen, beim Lösen des konstruierten Problems mit identischen Kapazitäten alle
Platzhalter zu akzeptieren.
Die Idee dabei ist es, die Platzhalterpfade so früh wie möglich zu akzeptieren, und erst danach die Originalpfade zu
betrachten.

Dazu wird eine Liste $L$ der Endknoten aller Pfade - jetzt mit Platzhalterpfaden - angelegt, wobei wir wieder zu jedem Knoten
eine Referenz auf den zugehörigen Pfad mitspeichern und damit entscheiden können, ob ein Anfangs- oder Zielknoten vorliegt.
Die Einträge der Liste sollen nach Knoten aufsteigend sortiert sein und bei identischen Knoten sollen
Zielknoten vor Anfangsknoten geordnet sein.
Jeder Knoten kommt also genau so oft vor, wie es Pfade mit ihm als einer der beiden Endknoten gibt.
Die gierige Ordnung $\leq_{G}$ erhalten wir, indem wir in der Liste jeden Zielknoten eines Pfades durch den Pfad selbst
ersetzen und die restlichen Knoten verwerfen.

Wir benutzen mit dem fiktiven Anführer und den $C$ virtuellen Pfaden wieder dieselbe Hilfspfade wie in
Abschnitt~\ref{subsec:algorithmusGleicheKapazitäten}, wobei diese in $\leq_G$ wieder vor den restlichen Pfaden stehen.
Für die Berechnung der bevorzugten Anführer jedes Pfades $p$ (also dem in $\leq_G$ größten Pfad $q = (s_q, t_q)$ mit
$t_q \leq s_p$) benötigen wir mit $\leq_G$ ebenfalls einen Zeitaufwand von
$\mathcal O(m)$.
Auch die Union-Find-Struktur bleibt die gleiche und jeder Pfad startet in seiner eigenen Gruppe.
Nun iteriert der Algorithmus durch die Liste $L$ der Endknoten und verarbeitet dabei Platzhalterpfade beim Antreffen vom
zugehörigen Anfangsknoten und Originalpfade beim Antreffen vom zugehörigen Zielknoten.
Die eigentliche Verarbeitung eines Pfades $p$ hat sich aber nicht geändert: Es wird der bevorzugte Anführer $q$ von $p$
betrachtet;
falls $\find(q)$ der fiktive Anführer ist, wird $p$ abgelehnt und die Gruppe von $p$ mit der
Gruppe des Vorgängers von $p$ vereinigt;
andernfalls wird p akzeptiert und in der Farbe von $\find(q)$
gefärbt und die Gruppe von $\find(q)$ mit der Gruppe des Vorgängers von $\find(q)$ vereinigt.

Das nächste Lemma soll nun zeigen, dass das Auffüllen mit Platzhaltern zusammen mit diesem Verfahren eine korrekte
Implementierung des gierigen Verfahrens darstellt:

\begin{lemma}
    Der beschriebene Algorithmus ist eine korrekte Implementierung des gierigen Verfahrens.
\end{lemma}
\begin{proof}
    Wir bezeichnen den oben beschriebenen Algorithmus nun mit $U$.
    Für $U$ haben wir wieder folgende Invariante für die Union-Find-Gruppen:
    Jede Gruppe mit verarbeiteten Pfaden enthält nur in $\leq_G$ aufeinanderfolgende Pfade und der Repräsentant einer
    solchen Gruppe ist entweder ein Anführer einer Farbe oder der fiktive Anführer.
    Andere Gruppen sind Einzelgruppen.

    Es ist leicht zu sehen, dass $U$ wie im vorherigen Abschnitt alle Pfade in $C$ Farben färbt, wobei sich zwei Pfade
    derselben Farbe nie schneiden können.
    Um zu zeigen, dass $U$ alle Platzhalterpfade färbt, nehmen wir an, $U$ würde einen Platzhalterpfad $p$ nicht färben.
    Sei $p$ derjenige Platzhalter mit kleinstem Anfangsknoten, der von $U$ nicht gefärbt wird, und sei $q$ sein bevorzugter Anführer.
    Das bedeutet, dass unmittelbar vor der Bearbeitung von $p$ der Repräsentant $\find(q)$ der fiktive Anführer war.
    Insbesondere gab es unter den $C$ gefärbten Anführern keinen optimalen Anführer von $p$, sondern alle gefärbten Anführer
    haben zu der Zeit einen Zielknoten, der rechts vom Anfangsknoten von $p$ liegt (sonst gäbe es einen optimalen Anführer).
    Da $p$ bereits am Anfangsknoten bearbeitet wird und alle Originalpfade erst am Zielknoten, kann kein Originalpfad einer
    dieser gefärbten Anführer sein, sondern es sind allesamt Platzhalterpfade, die bereits vor $p$ gefärbt worden sein müssen.
    Dementsprechend liegen auf der ersten Kante von $p$ mindestens $C+1$ Platzhalterpfade, was einen Widerspruch zur Konstruktion
    der Platzhalterpfade darstellt.

    Insbesondere berechnet $U$ wieder eine zulässige Teilmenge der Pfade.
    Jetzt müssen wir noch zeigen, dass genau die Originalpfade akzeptiert werden, die auch das gierige Verfahren
    akzeptiert.

    Dazu gehen wir vom Gegenteil aus und betrachten den ersten Originalpfad $p$ in gieriger Ordnung, bei dem die
    Entscheidung unterschiedlich ausgefallen ist.
    Da das gierige Verfahren immer eine minimale Lösung liefert, muss $p$ vom gierigen Verfahren akzeptiert und von
    $U$ abgelehnt worden sein, da die Menge, die $U$ berechnet, ebenfalls zulässig ist.
    Sei $A$ die Menge der Pfade, die unmittelbar vor der Bearbeitung von $p$ (von beiden Verfahren) akzeptiert wurden.
    Da der gierige Algorithmus zusätzlich $p$ akzeptiert hat, wissen wir, dass $A \cup \{p\}$ zulässig ist, was wir
    im Folgenden zum Widerspruch führen:
    Da $U$ den Pfad $p$ am Zielknoten verarbeitete (und $p$ damit der aktuell größte Pfad in gieriger Ordnung
    war), mussten sich zu dieser Zeit alle Anführer von Farben mit $p$ überschnitten haben.
    Dabei sei $l$ der in gieriger Ordnung kleinste Anführer einer Farbe und $e$ die letzte Kante seines Pfades.
    Von dieser wissen wir sicher, dass sie sich mit $p$ überschneidet.

	\begin{figure}[htbp]
		\centering
		\def\svgwidth{300bp}
		\input{proof_chain_greedy_arbitrary.pdf_tex}
		\caption{$c$-gefärbte Pfade sind schwarz. Mit einem Stern gekennzeichnete Pfade sind Anführer ihrer Farbe.
		Der durchgezogene Pfeil zeigt den tatsächlichen Verlauf der Farbe an. $p$ wurde vom gierigen Verfahren akzeptiert und von $U$ abgelehnt.}
		\label{fig:proof-chain-greedy-arbitrary}
	\end{figure}

    Angenommen, es existiere eine Farbe $c$, sodass kein $c$-gefärbter Pfad $e$ enthält.
    Da $l$ der kleinste Anführer ist, existiert ein $c$-gefärbter Pfad $p_1$ links von $e$ (z.B.\ der $c$-gefärbte virtuelle Pfad) und ein $c$-gefärbter Pfad $p_2$
    rechts von $e$ (z.B.\ der $c$-gefärbte Anführer).
    Wählt man, wie in Abbildung~\ref{fig:proof-chain-greedy-arbitrary} beispielhaft skizziert, diese jeweils möglichst nah an $e$, so erkennt man einen Widerspruch:
    Als $p_2$ mit $p_1$ als den scheinbar optimalen Anführer in $c$ gefärbt wurde, war tatsächlich $l$
    ein besserer Anführer von $p_2$, da sein Zielknoten weiter rechts ist.
    Insbesondere hätte $U$ $p_2$ also nicht in $c$ gefärbt.

    Daher besitzen alle $C$ Farben einen Pfad, der $e$ enthält, insgesamt gibt es also $C+1$ Pfade in $A \cup \{p\}$, die
    $e$ enthalten.
    Das ist allerdings ein Widerspruch dazu, dass $A \cup \{p\}$ eine zulässige Menge ist.
\end{proof}

Daraus folgt nun:

\begin{theorem}\label{theorem:greedyAlgorithm}
    Das gierige Verfahren berechnet eine Lösung für \CallControl\ in Kettennetzwerken mit beliebigen Kapazitäten und kann in einer Laufzeit von
    $\mathcal O(n+m)$ implementiert werden, wobei $n$ die Anzahl der Knoten der Kette und $m$ die Anzahl der gegebenen Pfade ist.
\end{theorem}

			
		\section{Optimale Lösung des \CallControl-Problems in Ringen}\label{sec:call-control-in-rings}
		Nachdem wir das Call-Control-Problem in Ketten optimal lösen können, betrachten wir nun das Problem in Ringen.
Sei also $(V, E)$ ein Ring mit $V=\{0,\dots,n-1\}$ und $E=\{e_0, \dots, e_{n-1}\}$, wobei $e_i$ die
Knoten $i$ und $(i+1 \Mod n)$ verbindet.
Außerdem haben wir eine Menge $P=\{p_1, \dots, p_m\}$ an Pfaden gegeben, unter denen jeder Pfad $p$ einen
Anfangsknoten $s_p$ mit einem Zielknoten $t_p$ verbindet mit $s_p \neq t_p$.
Dabei enthält der Pfad alle Kanten, auf die man trifft, wenn man sich vom Anfangsknoten im Uhrzeigersinn - das heißt
bis auf $\mathrm{mod}~n$ mit aufsteigenden Knotennamen - zum Zielknoten bewegt.
Alle Knoten eines Pfades bis auf den Anfangs- und Zielknoten heißen {\em innere Knoten}.

Wir können außerdem ohne Einschränkung von $n \leq 2m$ ausgehen, da wir solche Knoten entfernen können, die nicht
Anfangs- oder Zielknoten eines Pfades sind.
Dabei ersetzen wir die zwei Kanten, die am zu entfernenden Knoten anliegen, durch eine einzige Kante, dessen
Kapazität das Minimum der vorherigen beiden ist.

\begin{figure}[htbp]
    \centering
    \def\svgwidth{220bp}
    \input{example_ring.pdf_tex}
    \caption{Ein Ringnetzwerk mit 8 Kanten (innen) und 7 Pfaden (außen) (mod. nach~\cite{paper})}
    \label{fig:example-ring}
\end{figure}

Das Call-Control-Problem beschreibt hier wieder das Problem, eine möglichst große geeignete Teilmenge
$Q \subseteq P$ zu finden.
Dazu teilen wir die Menge der Pfade $P$ in zwei disjunkte Teilmengen $P_1$ und $P_2$.
$P_1$ bildet die Menge der Pfade in $P$, die den Knoten $0$ nicht als inneren Knoten haben; $P_2$
beinhaltet die restlichen Pfade, also $P \setminus P_1$.
Diese Partition können wir ganz einfach berechnen:
Wir sagen nun, dass jeder Pfad $p$, dessen Zielknoten $0$ ist, eigentlich den Zielknoten $n$ hat.
Ist dann $s_p < t_p$, so ist $0$ kein innerer Knoten von $p$ und $p$ wird $P_1$ zugeordnet, sonst $P_2$.

Abbildung~\ref{fig:example-ring} zeigt ein Beispiel eines Rings mit 8 Knoten sowie die Menge der sieben gegebenen Pfade (außen).
Dabei stellen die Pfade $p_1,\dots,p_5$ die Menge $P_1$ dar, $p_6$ und $p_7$ die Menge $P_2$.


Nun transformieren wir den Ring in eine Kette mit $2n$ Knoten:
Wir fertigen zwei Kopien der Kanten $e_1,\dotsc,e_n$ aus unserem Ring an und bilden eine Kette, indem wir die
beiden Kopien aneinanderhängen.
Die dazu gehörigen Knotennamen sind dabei $0, \dotsc, 2n-1$, wobei wir die Knoten $n, \dotsc, 2n-1$ durch
$0', \dotsc, (n-1)'$ als die zweite Kopie der Knoten kennzeichnen.
Die Pfade werden diesem Schema angepasst, sodass Pfade in $P_1$ nur Kanten der ersten Kopie belegen, wohingegen
jeder Pfad $p \in P_2$ aus zwei Teilpfaden besteht: Dem {\em Kopf}, der von $s_p$ bis nach $0'$ auf der ersten
Kopie der Kanten liegt, und dem {\em Schwanz}, der von $0'$ bis $t_p '$ auf der zweiten Kopie liegt.
In Abbildung~\ref{fig:example-ring-to-chain} wurde der Ring aus aus Abbildung~\ref{fig:example-ring} in die Kettenstruktur umgewandelt.

\begin{figure}[htbp]
    \centering
    \def\svgwidth{270bp}
    \input{example_ring_to_chain.pdf_tex}
    \caption{Die aus dem Ring in Abb.~\ref{fig:example-ring} konstruierte Kette (mod. nach~\cite{paper})}
    \label{fig:example-ring-to-chain}
\end{figure}

Wir definieren im Weiteren für eine Menge $Q$ von Pfaden die {\em Belastung} $L_1(Q, e_i)$ der Kante
$e_i$ in der ersten Kopie als die Anzahl der Pfade von $Q$, die die erste Kopie der Kante $e_i$ beinhalten.
Analog dazu sei $L_2(Q, e_i)$ die Belastung von $e_i$ in der zweiten Kopie.
\begin{definition}[Profil]
    Sei $Q$ eine Menge von Pfaden.
    Die monoton fallende Folge
    \[\pi_{Q}\coloneqq  (L_2(Q, e_0), \dots, L_2(Q, e_{n-1}))\]
    der Belastungen der $n$ Kanten in den zweiten Kopien heißt {\em Profil von $Q$}.
    Außerdem bezeichne $\pi_Q(e_i) \coloneqq L_2(Q, e_i)$ und für zwei Profile $\pi$ und $\pi'$ schreiben wir $\pi \leq \pi'$,
    falls $\pi(e_i) \leq \pi'(e_i)$ für alle $i \in \{0,\dots,n-1\}$.
\end{definition}
Wir bezeichnen eine Menge $Q$ von Pfaden im Ring als {\em kettengeeignet},
falls die Pfade von $Q$ in der oben konstruierten Kette keine Kapazitäten überschreiten, d.h.\ falls $L_1(Q, e) \leq c(e)$ und
$L_2(Q, e) \leq c(e)$ für alle Kanten $e$.
Gilt darüber hinaus $L_1(Q, e) + \pi(e) \leq c(e)$ für alle Kanten $e$, so heißt $Q$ {\em kettengeeignet zum Startprofil $\pi$}.
Anschaulich betrachtet, werden hier von den Kanten der ersten Kopie einige Kapazitäten bereits vom Startprofil belegt.
Betrachtet man jetzt nochmals die Konstruktion der Kette, kann man leicht erkennen, dass eine Menge $Q$ von Pfaden geeignet
(im Ring) ist genau dann, wenn $Q$ kettengeeignet zum eigens generierten Startprofil $\pi_Q$ ist, welches ja gerade den
Schwänzen der überstehenden Pfade entspricht.

\subsection{Der Algorithmus}

Nachdem wir nun alle notwendigen Vorbereitung getroffen haben, können wir mit dem eigentlichen Algorithmus beginnen.
Dabei gehen wir wie folgt vor:
Wir suchen mit binärer Suche nach dem größten $k \in \{0, \dots, m\}$, sodass eine geeignete Teilmenge $Q_k \subseteq P$
mit $k$ Pfaden existiert, und geben schließlich $Q_k$ aus.

Die Prozedur für das Finden einer geeigneten $k$-elementigen Menge von Pfaden bauen wir dabei wie folgt auf.
Wir starten mit dem leeren Profil $\pi_0$, das überall 0 ist.
In jeder Runde $i \in \set N$ initialisieren wir die Kapazitäten der beiden Kopien jeder Kante $e$ mit $c(e)$, wobei wir
in den ersten Kopie von $c(e)$ noch $\pi_{i-1}(e)$ Kapazitäten abziehen.
Das Profil $\pi_{i-1}$ belegt also bereits einige Kanten in der ersten Kopie und eine Menge $Q$ ist mit diesen Kapazitäten
kettengeeignet genau dann, wenn sie auf den ursprünglichen Kapazitäten kettengeeignet zum Startprofil $\pi_{i-1}$ ist.
Nun bestimmen wir mit dem gierigen Verfahren eine größtmögliche Menge $G$, die in der resultierenden Kette
geeignet ist.
Enthält $G$ weniger als $k$ Pfade, so bricht die Prodzedur ab mit der Antwort, dass eine geeignete $k$-elementige Menge
nicht existiert.
Ansonsten sei $G_i$ die Menge der ersten $k$ Pfade von $G$ in gieriger Ordnung und $\pi_{i}$ das dadurch erzeugte
Profil.
Ist nun $\pi_i = \pi_{i-1}$, so meldet die Prozedur einen Erfolg mit $G_i$ als Rückgabe, sonst wird die nächste Runde
berechnet.

Ein Beispiel kann man in \todo{FIGURE} sehen.

\subsection{Korrektheit des Algorithmus}\label{subsec:korrektheitCallControlInRings}
Um die Korrektheit des gesamten Algorithmus einzusehen, beweisen wir in den nächsten Lemmata, dass die Prozedur zum
Finden einer geeigneten $k$-elementigen Menge korrekt ist.
Dabei benutzen wir die vom Algorithmus generierten Mengen $G_i$, sowie dessen Profile $\pi_i$.

\begin{lemma}\label{lem:monotonousProfiles}
    Die Folge der Profile ist monoton wachsend, d.h.\ $\pi_i \leq \pi_{i+1}$ für alle $i \in \set N_0$.
\end{lemma}
\begin{proof}
    Durch Induktion: Für $i=0$ gilt die Aussage, da $\pi_0$ überall 0 ist.
    Angenommen, es gilt $\pi_{i-1} \leq \pi_i$.
    Nach Lemma~\ref{lem:optimalityGreedyAlgorithm} ist $G_{i}$ eine minimale Menge mit $k$ Pfaden, die auf der Kette mit den durch $\pi_{i}$
    reduzierten Kapazitäten geeignet ist.
    Da $G_{i-1}$ geeignet zum Startprofil $\pi_{i-1}$ ist und $\pi_{i-1} \leq \pi_i$, ist $G_{i-1}$ auch
    geeignet zum Startprofil $\pi_{i}$ und aufgrund der Minimalität von $G_i$ ist insbesondere $G_i \leq_G G_{i-1}$.
    Daraus folgt direkt $\pi_i \leq \pi_{i+1}$.
\end{proof}

Als Kondition dafür, dass unser Algorithmus hält, benötigen wir noch, dass die generierten Profile nicht
ewig wachsen können und, falls eine Lösung existiert, ab irgendeiner Runde konstant bleiben.

\begin{lemma}\label{lem:upperBoundProfiles}
    Falls eine im Ring geeignete Lösung $Q^*$ mit $k$ Pfaden existiert, dann ist $\pi_{Q^*}$ eine obere Grenze der
    generierten Profile $\pi_i$.
\end{lemma}
\begin{proof}
    Sei also $Q^*$ eine geeignete Lösung mit $k$ Pfaden.
    Wir zeigen wieder per Induktion für $i \in \set N_0$, dass $\pi_i \leq \pi_{Q^*}$.
    Für $i=0$ ist das offensichtlich, da $\pi_0$ überall 0 ist.
    Angenommen, $\pi_i \leq \pi_{Q^*}$.
    Wie wir oben gesehen haben, ist $Q^*$ kettengeeignet zum Startprofil $\pi_{Q^*}$, da es geeignet im Ring ist.
    Da $G_{i}$ mittels des gierigen Verfahren zum Startprofil $\pi_i$ ermittelt wird und $Q^*$ ebenfalls kettengeeignet
    zum Startprofil $\pi_i$ ist (da $\pi_i \leq \pi_{Q^*}$), gilt nach Lemma~\ref{lem:optimalityGreedyAlgorithm}, dass $G_{i} \leq_G Q^*$.
\end{proof}

Nun können wir die Korrektheit der Prozedur vervollständigen:

\begin{lemma}\label{lem:decisionProcedure}
    Die Prozedur liefert die korrekte Rückgabe in höchstens $n\cdot c(e_0)$ Runden.
\end{lemma}
\begin{proof}
    Für die Korrektheit unterscheiden wir, ob eine $k$-elementige geeignete Menge existiert.
    Nehmen wir an, es existiert keine solche Menge und der Algorithmus bricht nicht ab, sondern liefert eine Menge $Q_i$
    als Rückgabe.
    Dann müssen die letzten beiden Profile $\pi_{i-1}$ und $\pi_{i} = \pi_{Q_i}$ identisch gewesen sein.
    Da $Q_i$ aber durch das gierige Verfahren so berechnet wurde, dass es kettengeeignet zum Startprofil
    $\pi_{i-1}$ ist, ist es kettengeeignet zum eigenen Startprofil, insbesondere eine geeignete Menge im Ring, was im
    Widerspruch zur Annahme steht.

    Angenommen es existiere eine geeignete Menge $Q^*$ mit $k$ Elementen.
    Dann ist die Folge der generierten Profile $(\pi_i)_{i}$ nach Lemma~\ref{lem:upperBoundProfiles} durch $\pi_{Q^*}$
    nach oben beschränkt.
    Daher existiert zu jedem dieser Profile eine kettengeeignete $k$-elementige Menge (nämlich $Q^*$),
    weswegen das gierige Verfahren nach Lemma~\ref{lem:optimalityGreedyAlgorithm} eine mindestens $k$-elementige Menge zurückgibt.
    Daher kann der Algorithmus nicht ohne Rückgabe abbrechen.
    Da die Folge $(\pi_i)_i$ nach Lemma~\ref{lem:monotonousProfiles} außerdem monoton wachsend ist, konvergiert sie nach
    maximal $\sum_{j=0}^{n-1}\pi_{Q^*}(e_j)$ Runden.
    Also existiert ein minimales $i\in \set N$ mit $\pi_i = \pi_{i-1}$ und der Algorithmus gibt $Q_i$ zurück.
    Da $\pi_{i-1} = \pi_{Q_{i}}$, ist $Q_i$ kettengeeignet zum eigenen Startprofil und damit eine geeignete Menge.
    Weil ein Profil eine monoton fallende Folge ist, bekommen wir zusätzlich die Abschätzung von maximal
    $\sum_{j=0}^{n-1}\pi_{Q^*}(e_j) \leq n \cdot \pi_{Q^*}(e_0) \leq n \cdot c(e_0)$ Runden.
\end{proof}
Unsere Ergebnisse können wir nun im folgenden Theorem zusammenfassen:
\begin{theorem}
    Das Call-Control-Problem in Ringen kann in $\mathcal O(m \cdot n \cdot c_{\min} \cdot \log m)$ Zeit gelöst werden,
    wobei $n$ die Anzahl der Knoten, $m$ die Anzahl der Pfade und $c_{\min}$ die kleinste Kantenkapazität ist.
\end{theorem}
\begin{proof}
    Jede Runde des Entscheidungsalgorithmus ruft einmal das gierige Verfahren auf.
    Dieser hat nach Theorem~\ref{theorem:greedyAlgorithm} eine Laufzeit von $\mathcal O(n+m)$ und, da o.E.\
    $n \leq 2m$, gilt $\mathcal O(n+m) = \mathcal O(m)$.
    Nach Lemma~\ref{lem:decisionProcedure} können wir in $\mathcal O(n \cdot c(e_0))$ Runden eine geeignete $k$-elementige Menge finden.
    Dies kann auf $\mathcal O(n \cdot c_{\min})$ reduziert werden, indem die Kanten und Knoten des Rings so benannt werden, dass
    die Kante $e_0$ derjenigen mit der geringsten Kapazität entspricht (dies gelingt in $\mathcal \mathcal O(n+m)$ Zeit).
    Mit binärer Suche benötigen wir schließlich noch $\mathcal O(\log m)$ Entscheidungsrunden, um das größte $k \in \{0, \dots, m\}$
    mit einer geeigneten $k$-elementigen Menge zu finden.
\end{proof}














		
		\section{\WeightedCallControl\ in Ketten und Ringen}\label{sec:weighted-call-control}
		Nachdem wir nun optimale Lösungen für das \CallControl-Problem in Ketten und Ringen gefunden haben, modifizieren
wir in diesem Kapitel die Problemstellung ein wenig.
So geben wir jedem der Pfade $p$ ein bestimmtes Gewicht $\omega(p) \in \set R_+$ und versuchen statt der
Anzahl der akzeptierten Pfade $Q$ das Gesamtgewicht $\omega(Q) \colonequals \sum_{p\in Q}\omega(p)$ zu maximieren, ohne
dabei die Kapazität einer Kante zu verletzen.
Dabei verbraucht weiterhin jeder Pfad eine Kapazitätseinheit einer Kante, falls die Kante Teil des Pfades ist.

Wir werden dazu für das Problem in Ketten einen optimalen Algorithmus finden, jedoch für das Problem in Ringen
nur Approximationsalgorithmen kennenlernen. \todo{PTAS? BIPARTITE?}

\subsection{\WeightedCallControl-Problem in Ketten}\label{subsec:weighted-call-control-in-chains}
Für den Fall, dass alle Kanten $e$ wieder identische Kapazitäten $c(e) = C \in \set N$ haben, stellt \todo{REFERENZ CARLISLE}
eine optimale Lösung für das äquivalente Problem des $C$-Färbens von gewichteten Intervallen vor. \todo{check that we described why we can assemble these problems}.
Dabei wird das Problem auf ein Minimum-Cost Flow Problem reduziert, und dadurch eine Laufzeit von $O(C\cdot S(m))$
erreicht, wobei $m$ die Anzahl der Intervalle (hier: Pfade) und $S(m)$ die Laufzeit eines Kürzesten-Pfade-Algorithmus
in gerichteten Graphen mit positiven Kantengewichten und $m$ Kanten ist.
Außerdem wird angenommen, dass eine sortierte Liste aller Endpunkte bereits zur Verfügung steht. \todo{Sortieren :/}

Eine Instanz der Größe $C$ des Minimum-Cost Flow Problem besteht dabei aus einem gerichteten Graphen $(V,E)$ mit
einem Startknoten $s$ und Zielknoten $t$, wobei jede Kante $e$ des Graphen die Kosten $\cost(e)$ und die Kapazität
$\capacity(e) \geq 0$ besitzt.
Eine Funktion $f\colon E \rightarrow \set R$ heißt dabei {\em Fluss}, falls  $0 \leq f(e) \leq \capacity(e)$ für alle
Kanten $e$ und in jeden Knoten $v$ (ausgenommen $s$ und $t$) genauso viel hineinfließt wie auch wieder herausfließt, d.h.
die Summe des Flusses aller eingehenden Kanten von $v$ mit der der ausgehenden Kanten übereinstimmt.
Bei $s$ beträgt der ausgehende und bei $t$ der eingehende Fluss dabei jeweils $C$.
Gesucht ist nun ein Fluss, dessen Kosten $\sum_{e\in E}f(e) \cdot \cost(e)$ minimal sind.

Übertragen auf die Nomenklatur des Call-Control-Problems, skizzieren wir im Folgenden die Reduktion unseres Problems.
Sei $v_1 < \dots < v_r$ die sortierte Liste der Endpunkte aller Pfade ohne Duplikate.
Dann erstellen wir den gerichteten Graphen mit den Endpunkten $s = v_0, \dots, v_{r+1}=t$ und erstellen
für $1 \leq i \leq r+1$ eine Kante $(v_i-1, v_i)$ mit Kapazität $C$ und Kosten $0$.
Außerdem erstellen wir für jeden Pfad $p$ eine {\em Pfadkante} $(s_p, t_p)$ mit $s_p$ Start- und $t_p$ Zielknoten von $p$,
wobei die Kosten der Kante den negativen Gewichten des Pfades und die Kapazität der Kante $1$ entsprechen.
In \todo{FIGURE} ist eine solche Transformation erkennbar.
In einem Fluss $f$ repräsentieren dann die Pfadkanten $e$ mit $f(e)=1$ die zu akzeptiertenden Pfade.

Die Intuition hinter der Korrektheit dieser Problemformulierung besteht darin, den Fluss nicht als Ganzes, sondern
den Fluss der einzelnen $C$ Einheiten zu betrachten.
Dadurch erkennt man schnell, dass maximal $C$ Flusseinheiten sich überschneidende Pfade belegen können.
Für eine genauere Beschreibung und Analyse des Verfahren sowie dessen Laufzeit betrachte man \todo{referenz carlisle sec. 3}.
Wir wollen uns dagegen dem Problem mit willkürlichen Kapazitäten widmen.

\begin{theorem}
    Das gewichtete Call-Control-Problem einer Kette $(V, E)$ mit willkürlichen Kapazitäten kann in $\mathcal O(n + m\cdot S(m))$
    Zeitaufwand optimal gelöst werden.
    Dabei ist $n = |V|$, $m=|E|$ und $S(m)$ die Laufzeit eines Kürzesten-Pfade-Algorithmus in gerichteten Graphen mit
    positiven Kantengewichten und $m$ Kanten.
\end{theorem}
\begin{proof}
    Wir führen auch dieses Problem wieder auf ein Problem mit identischen Kapazitäten zurück.
    Dabei ist $C = \max_{e \in E}c(e)$ die maximale Kantenkapazität.
    Wie in Abschnitt~\ref{subsec:anpassenAnWillkürlicheKapazitäten} müssen die zusätzlichen $C - c(e)$ Kapazitäten
    aller Kanten $e$ durch Platzhalterpfade aufgefüllt werden.
    Erhalten wir eine optimale Lösung, in der alle Platzhalterpfade akzeptiert wurden, so ist die Menge der akzeptierten
    Originalpfade wieder eine optimale Lösung für das Ursprungsproblem.

    Um dafür zu sorgen, dass alle Platzhalterpfade $p$ in jedem Fall akzeptiert werden, setzen wir
    $\omega(p) \colonequals G + 1$, wobei $G$ die Summe der Gewichte aller Originalpfade ist.
    Nun hat der Algorithmus für identische Kapazitäten keine andere Wahl mehr, als jeden Platzhalterpfad zu akzeptieren,
    da jede Menge an Originalpfaden, die er stattdessen akzeptieren könnte, ein geringeres Gewicht hat und es nach
    Konstruktion genügend Kapazitäten gibt, um jeden Platzhalterpfad zu akzeptieren.

    Für die Sortierung der Endpunkte aller Pfade benötigen wir noch $\mathcal O(n+m)$ Zeit und erhalten insgesamt eine Laufzeit
    von $\mathcal O(n+m\cdot S(m))$.
\end{proof}

\subsection{Approximationsalgorithmus von Güte 2 für Ringe}
In diesem Abschnitt betrachten wir einen simplen Approximationsalgorithmus von Güte 2 für das Gewichtete
Call-Control-Problem in Ringen, d.h.\ einen Algorithmus der die optimale Lösung bis auf den Faktor 2 annähert.
Dabei gehen wir wie folgt vor:

Wir suchen die Kante $e$ des Rings mit der geringsten Kapazität $c_{\min}$ und betrachten die beiden Pfadmengen
$Q_{\overline e}^*$ und $Q_{e}^*$.
Dabei berechnen wir $Q_{\overline e}^*$ mit Hilfe von Abschnitt~\ref{subsec:weighted-call-control-in-chains} als die
optimale Lösung des Gewichteten Call-Control-Problem der Kette, die durch
Weglassen der Kante $e$ entsteht, mit allen Pfaden, die $e$ nicht enthalten.
Dagegen stellt $Q_e^*$ die Menge der $c_{\min}$ Pfade größten Gewichts, die $e$ enthalten, dar.
Zwischen $Q_{\overline e}^*$ und $Q_{e}^*$ wird nun die Menge mit größerem Gesamtgewicht als
Approximationslösung von Güte 2 gewählt.

\begin{theorem}
    Der vorangehende Algorithmus birgt eine Approximation von Güte 2 für das Gewichtete Call-Controll-Problem in Ringen.
\end{theorem}
\begin{proof}
    Es beschreibe $e$ die Kante aus dem Algorithmus mit der geringsten Kapazität.
    Zunächst sehen wir, dass $Q_{\overline e}^*$ und $Q_e^*$ geeignete Mengen darstellen.
    Das ist klar, da $Q_{\overline e}^*$ sogar eine geeignete Menge für die Kette ohne $e$ ist und $Q_e^*$ nur $c_{\min}$
    Pfade enthält, weshalb in beiden Fällen keine Kapazitäten verletzt werden.

    Sei $\OPT$ eine optimale Lösung für das Gewichtete Call-Control-Problem im Ring.
    $\OPT$ ist eine Partition der beiden Mengen $\OPT_e \colonequals \{p \in \OPT \mid p\ \text{enthält}\ e\}$ und
    $\OPT_{\overline e} \colonequals \{p\in\OPT\mid p\ \text{enthält}\ e\ \text{nicht}\}$.
    Es gilt insbesondere, dass $\omega(Q_{\overline e}^*) \geq \omega(\OPT_{\overline e})$ und
    $\omega(Q_e^*) \geq \omega(\OPT_{e})$, da $Q_{\overline e}$ und $Q_e$ in ihren Bereichen maximal gewählt wurden.
    Daher folgt für die berechnete Lösung $Q^*$:
    \[2 \cdot \omega(Q^*) = 2 \cdot  \max\{\omega(Q_e^*), \omega(Q_{\overline e}^*)\} \geq
    \omega(\OPT_e) + \omega(\OPT_{\overline e}) = \omega(\OPT) \]
\end{proof}

		
		\section{Fazit}
		Die vier Autoren des Artikels \glqq Call Control in Rings\grqq\ haben es geschafft, das \CallControl-Problem in Ketten mit
einem Algorithmus in linearer Zeit zu lösen.
Diesen Algorithmus konnte man dann beim Problem in Ringen elegant einsetzen, um so zu einem Algorithmus zu gelangen, der mit polynomiellem Zeitaufwand
eine optimale Lösung des \CallControl-Problems in Ringen berechnet.
Dies ist vor allem deswegen interessant, weil ein ähnlich wirkendes Problem, bei dem es darum geht, den größtmöglichen $k$-färbbaren
Teilgraphen in Kreisbogengraphen zu berechnen, {\em NP}-hart ist, wie in~\cite{circular-arc} nachgewiesen wurde.
Ein Kreisbogengraph $(V,E)$ mit Knotenmenge $V$ ist dabei wie folgt definiert:
Existiert eine Familie von Kreisbögen $(K_x)_{x\in V}$ um einen gemeinsamen Punkt mit festem Radius
und gilt $(x,y)\in E \Leftrightarrow K_x \cap K_y \neq \emptyset$, so ist $(V,E)$ ein Kreisbogengraph.

Für das \WeightedCallControl-Problem wurde anhand der Vorarbeit in~\cite{carlisle} ein effizienter Algorithmus in Ketten sowie ein darauf
aufbauender Approximationsalgorithmus von Güte 2 für das Problem in Ringnetzwerken entwickelt.
Außerdem zeigen die Autoren in~\cite{paper}, dass eine beliebig nahe Approximation in polynomieller Zeit gefunden werden kann.
Dazu wird ein sogenanntes Polynomial-Zeit-Approximations-Schema entwickelt, bei dem ein Parameter $\epsilon \in \set R_+$ gegeben ist
und der Algorithmus eine Lösung berechnet, dessen Gesamtgewicht mindestens das $(1-\epsilon)$-fache des Gesamtgewichts der optimalen Lösung ist.
Außerdem ist das \WeightedCallControl-Problem in Ringen sogar mindestens so schwer wie das Perfect-Matching-Problem in einem bipartiten Graphen (siehe \cite{hochbaum-levin}),
also einem Graphen, der in zwei unabhängige disjunkte Teilgraphen $U$ und $V$ geteilt werden kann, sodass jeder Knoten in $U$ mit mindestens einem Knoten in $V$ und andersherum verbunden ist.
Beim Perfect-Matching sucht man nach einer Teilmenge der Kanten, sodass jeder Knoten aus $U$ bzw.\ $V$ mit genau einem Knoten aus $V$ bzw.\ $U$ in Verbindung steht.
Jedoch ist noch immer unbekannt, ob dieses Problem in $P$ liegt oder {\em NP}-hart ist.

	\end{linenumbers}

    \bibliographystyle{plain}
    \bibliography{main}

\end{document}
