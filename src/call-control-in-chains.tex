Zunächst betrachten wir den einfacheren Fall des Problems, nämlich den in einer Kette:
Sei $(V,E)$ eine Kette mit Kapazitäten $c: E \to \set N$ und $V=\{0,\dots,n\}$, sodass für $i \in \{0,\dots,n-1\}$
eine Kante $e_i$ existiert, die die Knoten $i$ und $i+1$ verbindet.
Eine solche Kette nennen wir {\em $(0, n)$-durchnummeriert}.
\begin{definition}[Gierige Ordnung]
    Auf einer Menge $P$ von Pfaden in einer Kette $(V,E)$ nennen wir eine totale Ordnung $\leq_G$ {\em gierig},
    falls sie die Pfade nach ihren Endknoten aufsteigend ordnet, das heißt, falls
    $\forall p, q \in P \colon t_p < t_q \Rightarrow p <_G q$.
\end{definition}
Die Berechnung der Lösung erfolgt nun mit dem {\em gierigen Algorithmus}: Dieser arbeitet auf der Eingabe einer
Kette $(V,E)$ sowie einer Menge $A$ von Pfaden in $(V,E)$ in gieriger Ordnung sortiert.
Er bearbeitet alle Pfade in gieriger Ordnung und akzeptiert einen Pfad, falls dadurch keine der Kapazitäten
überschritten wird, ansonsten lehnt er den Pfad ab.
Nachdem alle Pfade bearbeitet wurden, endet der Algorithmus und liefert die akzeptierten Pfade als Rückgabe.
Im Folgenden bezeichne $\leq_G$ eine feste gierige Ordnung.
Jede geeignete Menge $A$ von $k$ Pfaden kann identifiziert werden durch eine Menge $\{a_1, \dots, a_k\}$ mit
$a_1 <_G \dots <_G a_k$.
Für $A=\{a_1,\dots,a_k\}$ und $B=\{b_1,\dots,b_k\}$ schreiben wir $A \leq_G B$, falls $\forall i \in \{1,\dots,k\}\colon a_i \leq_G b_i$.
Dabei nennen wir eine geeignete Menge $A$ {\em minimal}, falls für jede gleichmächtige, geeignete Menge $B$
gilt $A \leq_G B$.
\begin{lemma}[Optimalität des gierigen Algorithmus']
    \label{greedyAlgorithm}
    Existiert bei einer Menge $P$ von Pfaden einer Kette eine geeignete Teilmenge mit $k \in \set N$ Pfaden, so ist die
    Menge der ersten $k$ Pfade nach gieriger Ordnung, die der gierige Algorithmus berechnet, eine minimale Menge.
\end{lemma}
\begin{proof}
    Sei $Q_0$ eine geeignete Menge mit $k$ Pfaden.
    Wir nennen $G$ Menge der ersten $k$ Pfade nach gieriger Ordnung, die der gierige Algorithmus berechnet, und
    transformieren $Q_0$ schrittweise zu $Q_k = G$.
    Dabei soll für $i \in \{1,\dots\,k\}$ gelten, dass $Q_i$ geeignet ist sowie $Q_i \leq_G Q_{i-1}$.
    Dadurch ist auch $G$ geeignet und durch die Transitivität von $\leq_G$ auf Mengen von Pfaden folgt auch $G \leq_G Q_0$,
    also insbesondere die Minimalität von $G$.

    Als weitere Invariante nehmen wir auf, dass die ersten $i$ Pfade (in gieriger Ordnung) von $Q_i$ mit denen von $G$
    übereinstimmen.
    Für $i=0$ gelten alle Voraussetzungen.
    Nehmen wir also an, $Q_{i-1}$ sei geeignet und stimme auf den ersten $i-1$ Pfaden mit $G$ überein.
    Wir benennen $p$ als den $i$-ten Pfad von $G$.

    Für den Fall $p \in Q_{i-1}$ ist $p$ auch bereits der $i$-te Pfad von $Q_{i-1}$, da der Algorithmus nach
    gieriger Ordnung vorgeht und $Q_{i-1}$ mit $G$ auf den ersten $i-1$ Pfaden übereinstimmmt.
    Setzt man $Q_i \colonequals Q_{i-1}$ so bleiben alle Invarianten erhalten.

    Sonst gilt $p \notin Q_{i-1}$ und die Menge der Pfade $q \in Q_{i-1}$ mit $q >_G p$ ist nicht leer.
    Von diesen Pfaden sei $q$ ein solcher mit kleinstem Startknoten, der an $j$-ter Stelle in $Q_{i-1}$ steht
    ($j \geq i$).
    Setzen wir nun $Q_{i} \colonequals Q_{i-1} \setminus \{ q \} \cup \{ p \}$, so stimmen $Q_{i}$ und $G$ an den
    ersten $i$ Stellen wieder überein und alle Pfade, die in $Q_{i-1}$ an den Stellen $i$ bis $j-1$ stehen, rutschen
    in $Q_i$ eine Stelle weiter an die Positionen $i+1$ bis $j$.
    Insbesondere gilt also $Q_i \leq_G Q_{i-1}$.
    Bleibt zu zeigen, dass $Q_i$ wieder geeignet ist:
    Die Kanten, die links des Anfangsknoten $s_q$ von $q$ stehen, sind nicht betroffen, da aufgrund der Minimalität
    von $s_q$ dort nach $q$ keine weiteren Kapazitäten benötigt werden.
    Ist $s_q$ kleiner als der Anfangsknoten $s_p$ von $p$, so sparen die Kanten zwischen $s_q$ und $s_p$ sogar eine
    Kapazität ein.
    Ist andersrum $s_p$ kleiner als $s_q$, verletzt $Q_i$ auf den Zwischenkanten trotzdem keine Kapazitäten, da
    hier nach $p$ aufgrund der Minimalität von $s_q$ keine weiteren Pfade eine Kapazität verbrauchen und $p$ vom
    Algorithmus akzeptiert wurde, weil bis zum Pfad $p$ keine Kapazität verletzt wurde.
    Die Kanten, bei denen sich $p$ und $q$ überschneiden, spielen offensichtlich keine Rolle und die Kanten
    vom Endknoten von $p$ bis zum Endknoten von $q$ benötigen wieder eine Kapazität weniger.
\end{proof}

Daraus folgt, dass der gierige Algorithmus das Call-Control-Problem in Ketten löst:
Sei $Q$ eine Lösung, also eine möglichst große, geeignete Teilmenge von Pfaden in $P$, so liefert der gierige
Algorithmus nach Lemma~\ref{greedyAlgorithm} eine gleichmächtige, sogar minimale Lösung.

Eine einfache Implementierung dieses Algorithmus' könnte in $O(m \cdot n)$ , wobei $m$ die Anzahl der gegebenen Pfade
und $n$ die Anzahl der Knoten ist, indem einfach für jeden Pfad einzeln an jeder Kante überprüft wird, ob durch die
Hinzunahme Kapazitäten überschritten würden.
Jedoch lässt sich dieser Zeitaufwand sogar auf $O(m)$ verringern, wie wir in den nächsten Lemmata \todo{LEMMATA}
erkennen werden.
Dabei werden wir zunächst einen Algorithmus betrachten, bei dem vorausgesetzt wird, dass alle Kanten die gleiche
Kapazität haben, und werden diesen danach an unser Ausgangsproblem anpassen.

\subsection{Gieriger Algorithmus für gleiche Kapazitäten von Carlisle und Lloyd}\label{subsec:algorithmusGleicheKapazitäten}
Wir gehen davon aus, dass $(V,E)$ eine Kette ist, $P$ eine Menge von Pfaden in $(V,E)$ und jede Kante in $E$ nun eine
feste Kapazität $C \in \set N$ besitzt.
Die


























