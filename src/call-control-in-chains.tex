Zunächst betrachten wir den einfacheren Fall des Problems, nämlich den in einer Kette:
Sei $(V,E)$ eine Kette mit Kapazitäten $c: E \to \set N$ und $V=\{0,\dots,n\}$, sodass für $i \in \{0,\dots,n-1\}$
eine Kante $e_i$ existiert, die die Knoten $i$ und $i+1$ verbindet.
Eine solche Kette nennen wir \todo{todo} {\em $(0, n)$-durchnummeriert}.
\begin{definition}[Gierige Ordnung]
    Auf einer Menge $P$ von Pfaden in einer Kette $(V,E)$ nennen wir eine totale Ordnung $\leq_G$ {\em gierig},
    falls sie die Pfade nach ihren Endknoten aufsteigend ordnet, das heißt, falls
    $\forall p, q \in P \colon t_p < t_q \Rightarrow p <_G q$.
\end{definition}
Die Berechnung der Lösung erfolgt nun mit dem {\em gierigen Algorithmus}: Dieser arbeitet auf der Eingabe einer
Kette $(V,E)$ sowie einer Menge $A$ von Pfaden in $(V,E)$ in gieriger Ordnung sortiert. \todo{check sorting (p.229)}
Er bearbeitet alle Pfade in gieriger Ordnung und akzeptiert einen Pfad, falls dadurch keine der Kapazitäten
überschritten wird, ansonsten lehnt er den Pfad ab.
Nachdem alle Pfade bearbeitet wurden, endet der Algorithmus und liefert die akzeptierten Pfade als Rückgabe.
Im Folgenden bezeichne $\leq_G$ eine feste gierige Ordnung.
Jede geeignete Menge $A$ von $k$ Pfaden kann identifiziert werden durch eine Menge $\{a_1, \dots, a_k\}$ mit
$a_1 <_G \dots <_G a_k$.
Für $A=\{a_1,\dots,a_k\}$ und $B=\{b_1,\dots,b_k\}$ schreiben wir $A \leq_G B$, falls $\forall i \in \{1,\dots,k\}\colon a_i \leq_G b_i$.
Dabei nennen wir eine geeignete Menge $A$ {\em minimal}, falls für jede gleichmächtige, geeignete Menge $B$
gilt $A \leq_G B$.
\begin{lemma}[Optimalität des gierigen Algorithmus']
    \label{greedyAlgorithm}
    Existiert bei einer Menge $P$ von Pfaden einer Kette eine geeignete Teilmenge mit $k \in \set N$ Pfaden, so ist die
    Menge der ersten $k$ Pfade nach gieriger Ordnung, die der gierige Algorithmus berechnet, eine minimale Menge.
\end{lemma}
\begin{proof}
    Sei $Q_0$ eine geeignete Menge mit $k$ Pfaden.
    Wir nennen $G$ die Menge der ersten $k$ Pfade nach gieriger Ordnung, die der gierige Algorithmus berechnet, und
    transformieren $Q_0$ schrittweise zu $Q_k = G$.
    Dabei soll für $i \in \{1,\dots\,k\}$ gelten, dass $Q_i$ geeignet ist sowie $Q_i \leq_G Q_{i-1}$.
    Dadurch ist auch $G$ geeignet und durch die Transitivität von $\leq_G$ auf Mengen von Pfaden folgt auch $G \leq_G Q_0$,
    also insbesondere die Minimalität von $G$.

    Als weitere Invariante nehmen wir auf, dass die ersten $i$ Pfade (in gieriger Ordnung) von $Q_i$ mit denen von $G$
    übereinstimmen.
    Für $i=0$ gelten alle Voraussetzungen.
    Nehmen wir also an, $Q_{i-1}$ sei geeignet und stimme auf den ersten $i-1$ Pfaden mit $G$ überein.
    Wir benennen $p$ als den $i$-ten Pfad von $G$.

    Für den Fall $p \in Q_{i-1}$ ist $p$ auch bereits der $i$-te Pfad von $Q_{i-1}$, da der Algorithmus nach
    gieriger Ordnung vorgeht und $Q_{i-1}$ mit $G$ auf den ersten $i-1$ Pfaden übereinstimmmt.
    Setzt man $Q_i \colonequals Q_{i-1}$ so bleiben alle Invarianten erhalten.

    Sonst gilt $p \notin Q_{i-1}$ und die Menge der Pfade $q \in Q_{i-1}$ mit $q >_G p$ ist nicht leer.
    Von diesen Pfaden sei $q$ ein solcher mit kleinstem Startknoten, der an $j$-ter Stelle in $Q_{i-1}$ steht
    ($j \geq i$).
    Setzen wir nun $Q_{i} \colonequals Q_{i-1} \setminus \{ q \} \cup \{ p \}$, so stimmen $Q_{i}$ und $G$ an den
    ersten $i$ Stellen wieder überein und alle Pfade, die in $Q_{i-1}$ an den Stellen $i$ bis $j-1$ stehen, rutschen
    in $Q_i$ eine Stelle weiter an die Positionen $i+1$ bis $j$.
    Insbesondere gilt also $Q_i \leq_G Q_{i-1}$.
    Bleibt zu zeigen, dass $Q_i$ wieder geeignet ist:
    Die Kanten, die links des Anfangsknoten $s_q$ von $q$ stehen, sind nicht betroffen, da aufgrund der Minimalität
    von $s_q$ dort nach $q$ keine weiteren Kapazitäten benötigt werden.
    Ist $s_q$ kleiner als der Anfangsknoten $s_p$ von $p$, so sparen die Kanten zwischen $s_q$ und $s_p$ sogar eine
    Kapazität ein.
    Ist andersrum $s_p$ kleiner als $s_q$, verletzt $Q_i$ auf den Zwischenkanten trotzdem keine Kapazitäten, da
    hier nach $p$ aufgrund der Minimalität von $s_q$ keine weiteren Pfade eine Kapazität verbrauchen und $p$ vom
    Algorithmus akzeptiert wurde, weil bis zum Pfad $p$ keine Kapazität verletzt wurde.
    Die Kanten, bei denen sich $p$ und $q$ überschneiden, spielen offensichtlich keine Rolle und die Kanten
    vom Endknoten von $p$ bis zum Endknoten von $q$ benötigen wieder eine Kapazität weniger.
\end{proof}

Daraus folgt, dass der gierige Algorithmus das Call-Control-Problem in Ketten löst:
Sei $Q$ eine Lösung, also eine möglichst große, geeignete Teilmenge von Pfaden in $P$, so liefert der gierige
Algorithmus nach Lemma~\ref{greedyAlgorithm} eine gleichmächtige, sogar minimale Lösung.

Eine einfache Implementierung dieses Algorithmus' könnte in $O(m \cdot n)$ erfolgen, wobei $m$ die Anzahl der gegebenen
Pfade und $n$ die Anzahl der Knoten ist, indem einfach für jeden Pfad einzeln an jeder Kante überprüft wird, ob durch
die Hinzunahme die Kapazität überschritten würde.
Jedoch lässt sich dieser Zeitaufwand sogar auf $O(m)$ verringern, wie wir in den nächsten Lemmata \todo{LEMMATA}
erkennen werden.
Dabei werden wir zunächst einen Algorithmus betrachten, bei dem vorausgesetzt wird, dass alle Kanten die gleiche
Kapazität haben, und werden diesen danach an unser Ausgangsproblem anpassen.

\subsection{Gieriger Algorithmus für gleiche Kapazitäten von Carlisle und Lloyd}\label{subsec:algorithmusGleicheKapazitäten}
Wir gehen davon aus, dass $(V,E)$ eine Kette ist, $P$ eine Menge von $m$ Pfaden in $(V,E)$ und jede Kante in $E$ nun eine
feste Kapazität $C \in \set N$ besitzt.
Dabei können wir ohne Beschränkung der Allgemeinheit von $C \leq m$ ausgehen.
Die Idee des Algorithmus besteht darin, die akzeptierten Pfade nach gieriger Ordnung in $C$ verschiedene Farben zu
färben.

Dabei wird zu jeder Farbe der aktuelle {\em Anführer} gespeichert, das heißt, der in gieriger Ordnung größte, bereits
verarbeitete Pfad mit der Farbe.
Der Algorithmus verarbeitet alle Pfade in gieriger Ordnung und sucht für jeden Pfad $p$ den größten Anführer, der
sich nicht mit $p$ überschneidet.
Ein solcher Pfad heißt {\em optimaler Anführer von $p$}.
Findet er einen solchen, wird $p$ akzeptiert, in die Farbe seines optimalen Anführers gefärbt und damit neuer Anführer
der Farbe.
Findet er keinen, gibt es keine freie Farbe, das heißt das Akzeptieren des Pfades würde eine Kapazität verletzen und der
Pfad wird abgelehnt.
Eine solche Färbung mit $C = 2$, kann man in Abbildung \todo{ABBILDUNG} sehen.

\todo{FIGURE C-COLORING}

Speichert man allerdings zu jeder Farbe einfach den jeweiligen Anführer direkt ab, würde das Berechnen eines optimalen
Anführers für jeden Pfad mindestens $O(\log C)$ benötigen und damit würden wir das Ziel einer gesamt-linearen Laufzeit
$O(m)$ verpassen.

Stattdessen verwenden Carlisle und Lloyd in ihrem Algorithmus eine spezielle Union-Find-Instanz, die sog.
Static-Tree-Set-Union-Instanz, bei der die möglichen Vereinigungen bereits zu Beginn feststehen und in einem Baum
dargestellt werden können, wodurch \todo{check} $m$ find- und $n$ union-Aufrufe in einer Laufzeit von $O(m + n)$ erfolgen können.
\todo{Referenz A Linear-Time Algorithm for a Special Case of Disjoint Set Union HAROLD N. GABOW, On the k-coloring of intervals Martin C. Carlisle”, Errol L. Lloydb3*}

Genauer führt der Algorithmus zunächst $C$ weitere sogenannte {\em virtuelle Pfade} ein, die in gieriger Ordnung
vor den eigentlichen Pfaden $P$ stehen und die initialen Anführer der $C$ Farben sind.
In gieriger Ordnung an erster Stelle (noch vor den virtuellen Pfaden) wird ein weiterer {\em fiktiver Anführer $f$}
eingefügt, der die Darstellung nicht-akzeptierter Pfade ermöglicht.
Zu Beginn ist jeder Pfad in einer eigenen Gruppe der Union-Find-Struktur.
Der Repräsentant jeder Gruppe entspricht außerdem immer dem kleinsten Pfad (in gieriger Ordnung) innerhalb der Gruppe.
Im Laufe des Algorithmus' können die Gruppen mehrere in gieriger Ordnung aufeinanderfolgende Pfade beinhalten,
unverarbeitete Pfade bleiben vorerst in Einzelgruppen.
Wurde ein Pfad bereits verarbeitet, so ist der Repräsentant der Gruppe entweder ein Anführer einer Farbe oder der
fiktive Anführer $f$ (die virtuellen Pfade und der fiktive Anführer zählen bereits als verarbeitet).
Dementsprechend gibt es zu jeder Zeit unter den verarbeiteten Pfaden genau $C+1$ Gruppen, wobei die Gruppe des fiktiven
Anführers immer die \glqq kleinste\grqq{} Gruppe bleibt.
Außerdem wird zu jedem Pfad gespeichert, ob er \todo{check what's really necessary} gerade Anführer ist, welche Farbe er
hat und ob er schon bearbeitet wurde.

Nun berechnet der Algorithmus zu jedem Pfad $p \in P$ den {\em bevorzugten Anführer von $p$}, also den größten Pfad in
gieriger Ordnung, dessen Endknoten kleinergleich dem Anfangsknoten von $p$ ist.
\todo{check this:}
Das kann man mit einfachem Durchlaufen aller Endpunkte in $P$ in aufsteigender Reihenfolge erreichen, also in $O(n)$
Zeit.
Es ist offensichtlich, dass kein optimaler Anführer von $p$ in gieriger Ordnung größer als der bevorzugte Anführer von
$p$ sein kann.

Der Algorithmus bearbeitet danach jeden Pfad $p \in P$ nach gieriger Reihenfolge, wobei er jeweils wie folgt vorgeht.
Es wird der bevorzugte Pfad $q$ von $p$ betrachtet:
Ist $\find(q)$, also der Repräsentant der Gruppe von $q$, der fiktive Anführer $f$, so gibt es keinen optimalen Anführer
zu $p$, da auch alle Pfade in $P$, die kleinergleich dem bevorzugten Pfad $q$ und damit Kandidaten für einen optimalen
Anführer von $p$ wären, auch in der Gruppe von $f$ liegen und damit kein Anführer einer Farbe dabei ist, da diese die
Repräsentanten der anderen C Gruppen sind.
Daher wird $p$ abgelehnt und die Gruppe von $p$ mit der Gruppe des Vorgängers von $p$ in gieriger Ordnung vereinigt.
Die Invarianten bleiben erhalten, da wieder $C+1$ Gruppen unter den verarbeiteten Pfaden existieren mit den gleichen
Anführern.

Ist $\find(q)$ jedoch ein Anführer einer Farbe, so ist es mit Sicherheit der optimale Anführer von $p$, da es der größte
Anführer ist, der noch kleinergleich dem bevorzugten Anführer von $p$ ist (da die Pfade in den Gruppen
aufeinanderfolgend sind).
Also wird $p$ akzeptiert, in dieser Farbe gefärbt und damit neuer Anführer der Farbe.
Da $\find(q)$ nun kein Anführer mehr ist, wird seine Gruppe aufgelöst und mit der Gruppe des Vorgängers von $\find(q)$
vereinigt.
Wieder erhalten wir alle Invarianten.

\subsection{Anpassen an willkürliche Kapazitäten}
















