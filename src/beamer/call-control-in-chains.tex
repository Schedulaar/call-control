\begin{section}{Call-Control in Ketten}

\subsection{Das gierige Verfahren}

\begin{frame}{Gierige Ordnung}
	\begin{definition}[Gierige Ordnung]
		Auf einer Menge $P$ von Pfaden in einer Kette nennen wir eine Totalordnung $\leq_G$ mit zugehöriger strenger Totalordnung
		$<_G$ {\em gierig},
		falls $\forall p, q \in P \colon t_p < t_q \Rightarrow p <_G q$.
	\end{definition}
	\only<1>{
		\begin{figure}[htbp]
			\centering
			\small
			\def\svgwidth{200bp}
			\input{greedy-order-chain-1.pdf_tex}
	\end{figure}}
	\only<2>{
		\begin{figure}[htbp]
		\centering
		\small
		\def\svgwidth{200bp}
		\input{greedy-order-chain-2.pdf_tex}
		\end{figure}}
\end{frame}

\begin{frame}[t]{Das gierige Verfahren}
	Eine gierige Ordnung $\leq_G$ ist bereits gegeben.
	\only<1>{
		\begin{figure}[htbp]
			\centering
			\small
			\def\svgwidth{200bp}
			\input{greedy-algorithm-with-example-chain-1.pdf_tex}
	\end{figure}}
	\only<2>{
		\begin{figure}[htbp]
			\centering
			\small
			\def\svgwidth{200bp}
			\input{greedy-algorithm-with-example-chain-2.pdf_tex}
		\end{figure}}
	\only<3>{
		\begin{figure}[htbp]
			\centering
			\small
			\def\svgwidth{200bp}
			\input{greedy-algorithm-with-example-chain-3.pdf_tex}
		\end{figure}}
	\only<4>{
		\begin{figure}[htbp]
			\centering
			\small
			\def\svgwidth{200bp}
			\input{greedy-algorithm-with-example-chain-4.pdf_tex}
		\end{figure}}
	\only<5>{
		\begin{figure}[htbp]
			\centering
			\small
			\def\svgwidth{200bp}
			\input{greedy-algorithm-with-example-chain-5.pdf_tex}
		\end{figure}}
	\only<6>{
		\begin{figure}[htbp]
			\centering
			\small
			\def\svgwidth{200bp}
			\input{greedy-algorithm-with-example-chain-6.pdf_tex}
		\end{figure}}
	\only<7-9>{
		\begin{figure}[htbp]
			\centering
			\small
			\def\svgwidth{200bp}
			\input{greedy-algorithm-with-example-chain-7.pdf_tex}
		\end{figure}
		\begin{itemize}
			\item<8-> Menge der akzeptierten Pfade ist optimale Lösung.
			\item<9> Einfache Implementierung in $\mathcal{O}(m\cdot n)$ Zeit möglich ($m$ Anzahl Pfade, $n$ Anzahl Knoten).
		\end{itemize}
	}
\end{frame}

\begin{frame}{Korrektheit des gierigen Verfahrens -- I}
	\begin{figure}[htbp]
		\centering
		\small
		\def\svgwidth{180bp}
		\input{greedy-order-chain-2.pdf_tex}
	\end{figure}
	\begin{itemize}
		\item Seien $A=\{a_1,\dots,a_k\}$ und $B=\{b_1,\dots,b_k\}$ Teilmengen der Pfade $P$ mit $a_1 \leq_G \dots \leq_G a_k$ und $b_1 \leq_G \dots \leq_G b_k$. Wir schreiben $A \leq_G B$, falls $\forall i \leq k\colon a_k \leq_G b_k$.
		\pause
		\item Bsp.: $\{p_1, p_3, p_6\} \leq_G \{p_1, p_4, p_6\}$.
		\pause
		\item Eine zulässige Menge $A$ heißt minimal, falls $A \leq_G B$ für alle zulässigen Mengen $B$ mit $|A|=|B|$. 
	\end{itemize}
\end{frame}

\begin{frame}{Korrektheit des gierigen Verfahrens -- II}
\begin{lemma}[Optimalität des gierigen Verfahrens]\label{lem:optimalityGreedyAlgorithm}
	Existiert eine zulässige Teilmenge $Q_0$ mit $k \in \set N$ Pfaden, so ist die
	Menge $G$ der in gieriger Ordnung $\leq_G$ kleinsten $k$ Pfade, die das gierige Verfahren berechnet, eine minimale Menge.
\end{lemma}
\pause
Beweisskizze:
\begin{itemize}
	\item Transformiere $Q_0$ in $k$ Schritten in $G$ und erhalte:\\ $Q_i$ zulässig, $Q_{i+1} \leq_G Q_i$ und $Q_i$ stimmt auf ersten $i$ Pfaden mit $G$ überein.
	\pause
	\item I.S.: $p$ sei $i$-ter Pfad von $G$ \pause mit $p \notin Q_{i-1}$. 
		
		\pause $q$ sei Pfad aus $Q_{i-1}$ mit $q >_G p$ und kleinstem Startknoten.
		
		\pause Erhalte $Q_i$ durch Ersetzen von $q$ durch $p$ in $Q_{i-1}$.
		
		\pause Dann $Q_i \leq_G Q_{i-1}$. \pause Mit I.V.: $Q_i$ zulässig, da keine Kantenkapazität verletzt wird.
\end{itemize}
\end{frame}
	
\subsection{Identische Kapazitäten}
\begin{frame}[t]{Algorithmus für identische Kapazitäten}
	\begin{itemize}
		\item Feste Kapazität $C \in \set N$ für alle Kanten.
		\pause \item Beispiel mit $C=2$:
	\end{itemize}

	\only<2>{\begin{figure}[htbp]
		\centering
		\def\svgwidth{200bp}
		\input{k-coloring-1.pdf_tex}
	\end{figure}}

	\only<3->{\begin{figure}[htbp]
		\centering
		\def\svgwidth{200bp}
		\input{k-coloring-2.pdf_tex}
	\end{figure}}

	\only<4>{\begin{itemize}
			\item Call-Control-Problem entspricht maximaler $C$-Färbung.
		\end{itemize}}
\end{frame}

\begin{frame}{Algorithmus für $C$-Färbung}
		
	\begin{itemize}
		\pause\item Füge virtuelle Pfade $v_1,\dots,v_C$, je unterschiedlich gefärbt in einer der $C$ Farben, vor allen Pfaden ein.
		\pause\item Speichere zu jeder Farbe $c$ den {\em aktuellen Anführer von $c$} (den in $\leq_G$ größten $c$-gefärbten Pfad). Zu Beginn: Der zugehörige virtuelle Pfad.
		\begin{figure}[htbp]
			\centering
			\tiny
			\def\svgwidth{200bp}
			\input{k-coloring-with-virtuals.pdf_tex}
		\end{figure}
		\pause\item Suche bei Bearbeitung von Pfad $p$ den {\em optimalen Anführer von $p$}, d.h.\ den aktuell größten Anführer, der sich nicht mit $p$ überschneidet.
		\pause\item Ist das möglich in gesamt-linearer Zeit?
	\end{itemize}

\end{frame}

\begin{frame}{$C$-Färbung -- Union-Find-Algorithmus}
	\begin{itemize}
		\pause\item Füge weiteren Pfad $f$, den fiktiven Anführer, als ersten Pfad ein.
		\pause\item Ermittle für jeden Pfad $p$ seinen {\em bevorzugten Anführer}, d.h.\ den in $\leq_G$ größten Pfad, dessen Zielknoten nicht nach dem Anfangsknoten $s_p$ kommt.
		\pause\begin{figure}[htbp]
			\centering
			\small
			\def\svgwidth{180bp}
			\input{union-find-setup.pdf_tex}
		\end{figure}
		\pause\item Erstelle Union-Find-Instanz mit jedem Pfad in eigener Gruppe.
	\end{itemize}
\end{frame}

\begin{frame}{$C$-Färbung -- Union-Find-Algorithmus}
\begin{itemize}
	\item Invarianten:
	\begin{itemize}
		\pause\item Repräsent einer Gruppe ist in $\leq_G$ kleinster Pfad der Gruppe.
		\pause\item Gruppen enthalten nur in $\leq_G$ aufeinanderfolgende Pfade
		\pause\item Nicht verarbeitete Pfade sind in Einzelgruppen; Repräsentant einer verarbeiteten Gruppe ist Anführer einer Farbe oder der fiktive Anführer.
	\end{itemize}
	\pause\item Bei Bearbeitung von Pfad $p$ mit bevorzugtem Anführer $q$:
	\begin{itemize}
		\pause\item $\find(q)=f$: Verwerfe $p$ und vereinige die Gruppe von $p$ mit der des Vorgängers von $p$.
		\pause\item $\find(q)$ ist $c$-gefärbter Anführer: Akzeptiere $p$, färbe $p$ in $c$ und vereinige die Gruppe von $\find(q)$ mit der des Vorgängers von $\find(q)$.
	\end{itemize}
\end{itemize}
\end{frame}

\begin{frame}{$C$-Färbung -- Union-Find-Algorithmus am Beispiel}
	\only<1>{\begin{figure}[htbp]
			\centering
			\small
			\def\svgwidth{200bp}
			\input{union-find-algo-1.pdf_tex}
		\end{figure}
		\begin{figure}[htbp]
			\centering
			\small
			\def\svgwidth{250bp}
			\input{union-find-structure-1.pdf_tex}
		\end{figure}}
	\only<2>{\begin{figure}[htbp]
			\centering
			\small
			\def\svgwidth{200bp}
			\input{union-find-algo-2.pdf_tex}
		\end{figure}
		\begin{figure}[htbp]
			\centering
			\small
			\def\svgwidth{250bp}
			\input{union-find-structure-2.pdf_tex}
		\end{figure}}	
	\only<3>{\begin{figure}[htbp]
			\centering
			\small
			\def\svgwidth{200bp}
			\input{union-find-algo-3.pdf_tex}
		\end{figure}
		\begin{figure}[htbp]
			\centering
			\small
			\def\svgwidth{250bp}
			\input{union-find-structure-3.pdf_tex}
		\end{figure}}	
	\only<4>{\begin{figure}[htbp]
			\centering
			\small
			\def\svgwidth{200bp}
			\input{union-find-algo-4.pdf_tex}
		\end{figure}
		\begin{figure}[htbp]
			\centering
			\small
			\def\svgwidth{250bp}
			\input{union-find-structure-4.pdf_tex}
		\end{figure}}	
	\only<5>{\begin{figure}[htbp]
			\centering
			\small
			\def\svgwidth{200bp}
			\input{union-find-algo-5.pdf_tex}
		\end{figure}
		\begin{figure}[htbp]
			\centering
			\small
			\def\svgwidth{250bp}
			\input{union-find-structure-5.pdf_tex}
		\end{figure}}	
	\only<6>{\begin{figure}[htbp]
			\centering
			\small
			\def\svgwidth{200bp}
			\input{union-find-algo-6.pdf_tex}
		\end{figure}
		\begin{figure}[htbp]
			\centering
			\small
			\def\svgwidth{250bp}
			\input{union-find-structure-6.pdf_tex}
		\end{figure}}	
	\only<7>{\begin{figure}[htbp]
			\centering
			\small
			\def\svgwidth{200bp}
			\input{union-find-algo-7.pdf_tex}
		\end{figure}
		\begin{figure}[htbp]
			\centering
			\small
			\def\svgwidth{250bp}
			\input{union-find-structure-7.pdf_tex}
	\end{figure}}
\end{frame}

\begin{frame}{$C$-Färbung -- Union-Find-Algorithmus}
	\begin{itemize}
		\pause\item Wir benötigen $m$ find- und union-Aufrufe.
		\pause\item Alle Vereinigungen geschehen entlang einer Kette.
		\pause\item Mit Static-Tree-Set-Union benötigen wir $\mathcal{O}(m)$ Zeit dafür (Gabow und Tarjan in [3]).
		\pause\item Das Call-Control-Problem mit identischen Kapazitäten ist in $\mathcal{O}(m)$ Zeit optimal lösbar, wenn die Menge der Pfade bereits in gieriger Ordnung sortiert ist.
		\pause\item Genauere Analyse des Verfahrens durch Carlisle und Lloyd in [2].
	\end{itemize}
\end{frame}

\subsection{Willkürliche Kapazitäten}
\begin{frame}{Anpassen für willkürliche Kapazitäten}
	\begin{itemize}
		\pause\item Betrachten willkürliche Kapazitäten $c: E \rightarrow \set N$.
		\pause\item Anpassung der Idee für identische Kapazitäten:
		\begin{itemize}
			\pause\item Setze $C \colonequals \max_{e \in E} c(e)$ als neue Kapazität jeder Kante.
			\pause\item Füge an Kanten mit überflüssigen Kapazitäten Platzhalterpfade ein:
				\begin{figure}[htbp]
					\centering
					\small
					\def\svgwidth{200bp}
					\input{dummy_paths.pdf_tex}
				\end{figure}
			\pause\item Sorge dafür, dass alle Platzhalterpfade akzeptiert werden.
		\end{itemize}
		\pause\item Probleme: Anzahl Platzhalter? Wie akzeptieren wir alle Platzhalter?
	\end{itemize}
\end{frame}
\begin{frame}{Anzahl der Platzhalterpfade}
	\begin{itemize}
		\pause\item Für bestimmte Kettennetzwerke kann es passieren, dass wir $\Omega(n\cdot m)$ Platzhalter einfügen:
		\begin{figure}[htbp]
			\centering
			\small
			\def\svgwidth{180bp}
			\input{dummy_paths_too_many.pdf_tex}
		\end{figure}
		\pause\item Flache die Kapazitäten ab mit 
		\[
		c'(e_i) =
		\begin{cases}
		\min(c(e_0), n_0) &\quad\text{für}\ i=0\\
		\min(c(e_i), c'(e_{i-1}) + n_i) &\quad\text{für}\ i \geq 1
		\end{cases}
		\]
		wobei $n_i$ die Anzahl der Pfade in $P$ mit Anfangsknoten $i$ ist.
		\pause\item Damit werden nur $\mathcal{O}(m)$ Platzhalter generiert.
		\pause\item Anpassen der Kapazitäten und Auffüllen mit Platzhaltern in $\mathcal{O}(n+m)$ Zeit möglich.
	\end{itemize}
\end{frame}

\begin{frame}{Akzeptieren der Platzhalter}
	\pause Ähnliches Vorgehen wie zuvor, versuche nun aber Platzhalterpfade möglichst früh zu bearbeiten:
	\begin{itemize}
		\pause\item Erstelle eine Liste $L$ der Endknoten aller Pfade (zu jedem Eintrag der Liste speichere eine Referenz auf zugehörigen Pfad):
		\pause\begin{itemize}
			\item Nach Endknoten aufsteigend sortiert
			\item Bei gleichem Endknoten sollen Anfangsknoten vor Zielknoten geordnet werden
		\end{itemize}
		\pause\item Erhalte $\leq_G$ durch Ersetzen von Zielknoten der Liste durch die zug.\ Pfade.
		\pause\item Füge wieder $C$ virtuelle Pfade sowie den fiktiven Anführer vor den anderen Pfaden ein und ordne bevorzugte Anführer zu.
		\pause\item Statt die Pfade in der Reihenfolge von $\leq_G$ zu bearbeiten wird nun die Liste $L$ durchlaufen und
		\begin{itemize}
			\item Platzhalterpfade bei Antreffen ihres Anfangsknotens
			\item Originalpfade bei Antreffen ihres Zielknotens
		\end{itemize}
	 wie im vorherigen Algorithmus verarbeitet.
	\end{itemize}
\end{frame}

\begin{frame}{Korrektheit des Algorithmus I}
	\begin{lemma}[Korrektheit des Algorithmus]
		Der beschriebene Algorithmus $U$ ist eine korrekte Implementierung des gierigen Verfahrens $G$ für willkürliche Kapazitäten.
	\end{lemma}
	Beweisskizze: 
	\begin{itemize}
		\pause\item $U$ berechnet wieder $C$-Färbung der Pfade mit Platzhalter.
		\pause\item $U$ akzeptiert alle Platzhalterpfade ($U$ berechnet insb. eine zulässige Menge).
		\pause\item $U$ akzeptiert alle Pfade, die das gierige Verfahren akzeptiert.
	\end{itemize}
\end{frame}

\begin{frame}{Korrektheit des Algorithmus II}
	$U$ akzeptiert alle Pfade, die das gierige Verfahren $G$ akzeptiert.\\ Über Widerspruch:
	\begin{itemize}
		\pause\item $p$ erster Originalpfad mit unterschiedlicher Entscheidung.
		\begin{itemize}
			\pause\item $A$ sei die Menge der (gemeinsam) akzeptierten Pfade vor $p$.
			\pause\item Dann: $U$ verwirft $p$, $G$ akzeptiert $p$. Insb.\ $A \cup \{p\}$ zulässig.
		\end{itemize} 
		\pause\item Sei $l$ kleinster Anführer einer Farbe zur Zeit der Bearbeitung von $p$, $e$ die letzte Kante von $l$.
		\pause\item $\mathbb{A}: \exists c\ \text{Farbe}$, sodass kein $c$-gefärbter Pfad $e$ enthält.
			\vspace{-1em}
			\begin{columns}[T]
				\begin{column}{0.07\textwidth}
				\end{column}
				\begin{column}{0.45\textwidth}
					\begin{itemize}
						\item Es ex. $c$-gefärbte Pfade links und rechts von $e$.
						\vspace{-0.5em}
						\item $p_1$, $p_2$ seien solche möglichst nah an $e$.
						\vspace{-0.3em}
						\item Widerspruch: $l$ besserer Anführer von $p_2$ als $p_1$.
					\end{itemize}
				\end{column}
				\begin{column}{0.5\textwidth}
					\begin{figure}[htbp]
						\tiny
						\def\svgwidth{\textwidth}
						\input{proof_chain_greedy_arbitrary.pdf_tex}
					\end{figure}
				\end{column}
			\end{columns}
		\pause\item Also gibt es (zusätzlich zu $p$) $C$ weitere von $G$ akzeptierte Pfade, die $e$ beinhalten.
	\end{itemize}
\end{frame}

\begin{frame}{Rekapitulation}
	Was haben wir erreicht?
	\begin{itemize}
		\pause\item Das gierige Verfahren löst das Call-Control-Problem für willkürliche Kapazitäten optimal.
		\pause\item Mit einer speziellen Union-Find-Struktur finden wir eine Implementierung für identische Kapazitäten mit Laufzeit $\mathcal{O}(m)$.
		\pause\item Wir verwenden dann eine angepasste Version mit Platzhalterpfaden, um willkürliche Kapazitäten zu erlauben, und erhalten:
	\end{itemize}
	\pause\begin{theorem}
  		Das gierige Verfahren berechnet eine optimale Lösung für \CallControl\ in Ketten mit willkürlichen Kapazitäten und kann in einer Laufzeit von
		$\mathcal O(n+m)$ implementiert werden.
	\end{theorem}
\end{frame}

\end{section}