\begin{section}{Call-Control in Ketten}

\subsection{Das gierige Verfahren}

\begin{frame}{Gierige Ordnung}
	\begin{definition}[Gierige Ordnung]
		Auf einer Menge $P$ von Pfaden in einer Kette nennen wir eine Totalordnung $\leq_G$ mit zugehöriger strenger Totalordnung
		$<_G$ {\em gierig},
		falls $\forall p, q \in P \colon t_p < t_q \Rightarrow p <_G q$.
	\end{definition}
	\only<1>{
		\begin{figure}[htbp]
			\centering
			\small
			\def\svgwidth{200bp}
			\input{greedy-order-chain-1.pdf_tex}
	\end{figure}}
	\only<2>{
		\begin{figure}[htbp]
		\centering
		\small
		\def\svgwidth{200bp}
		\input{greedy-order-chain-2.pdf_tex}
		\end{figure}}
\end{frame}

\begin{frame}[t]{Das gierige Verfahren}
	Eine gierige Ordnung $\leq_G$ ist bereits gegeben.
	\only<1>{
		\begin{figure}[htbp]
			\centering
			\small
			\def\svgwidth{200bp}
			\input{greedy-algorithm-with-example-chain-1.pdf_tex}
	\end{figure}}
	\only<2>{
		\begin{figure}[htbp]
			\centering
			\small
			\def\svgwidth{200bp}
			\input{greedy-algorithm-with-example-chain-2.pdf_tex}
		\end{figure}}
	\only<3>{
		\begin{figure}[htbp]
			\centering
			\small
			\def\svgwidth{200bp}
			\input{greedy-algorithm-with-example-chain-3.pdf_tex}
		\end{figure}}
	\only<4>{
		\begin{figure}[htbp]
			\centering
			\small
			\def\svgwidth{200bp}
			\input{greedy-algorithm-with-example-chain-4.pdf_tex}
		\end{figure}}
	\only<5>{
		\begin{figure}[htbp]
			\centering
			\small
			\def\svgwidth{200bp}
			\input{greedy-algorithm-with-example-chain-5.pdf_tex}
		\end{figure}}
	\only<6>{
		\begin{figure}[htbp]
			\centering
			\small
			\def\svgwidth{200bp}
			\input{greedy-algorithm-with-example-chain-6.pdf_tex}
		\end{figure}}
	\only<7-9>{
		\begin{figure}[htbp]
			\centering
			\small
			\def\svgwidth{200bp}
			\input{greedy-algorithm-with-example-chain-7.pdf_tex}
		\end{figure}
		\begin{itemize}
			\item<8-> Menge der akzeptierten Pfade ist optimale Lösung.
			\item<9> Einfache Implementierung in $\mathcal{O}(m\cdot n)$ Zeit möglich ($m$ Anzahl Pfade, $n$ Anzahl Knoten).
		\end{itemize}
	}
\end{frame}

\begin{frame}{Korrektheit des gierigen Verfahrens -- I}
	\begin{figure}[htbp]
		\centering
		\small
		\def\svgwidth{180bp}
		\input{greedy-order-chain-2.pdf_tex}
	\end{figure}
	\begin{itemize}
		\item Seien $A=\{a_1,\dots,a_k\}$ und $B=\{b_1,\dots,b_k\}$ Teilmengen der Pfade $P$ mit $a_1 \leq_G \dots \leq_G a_k$ und $b_1 \leq_G \dots \leq_G b_k$. Wir schreiben $A \leq_G B$, falls $\forall i \leq k\colon a_k \leq_G b_k$.
		\pause
		\item Bsp.: $\{p_1, p_3, p_6\} \leq_G \{p_1, p_4, p_6\}$.
		\pause
		\item Eine zulässige Menge $A$ heißt minimal, falls $A \leq_G B$ für alle zulässigen Mengen $B$ mit $|A|=|B|$. 
	\end{itemize}
\end{frame}

\begin{frame}{Korrektheit des gierigen Verfahrens -- II}
\begin{lemma}[Optimalität des gierigen Verfahrens]\label{lem:optimalityGreedyAlgorithm}
	Existiert eine zulässige Teilmenge $Q_0$ mit $k \in \set N$ Pfaden, so ist die
	Menge $G$ der in gieriger Ordnung $\leq_G$ kleinsten $k$ Pfade, die das gierige Verfahren berechnet, eine minimale Menge.
\end{lemma}
\pause
Beweisskizze:
\begin{itemize}
	\item Transformiere $Q_0$ in $k$ Schritten in $G$ und erhalte:\\ $Q_i$ zulässig, $Q_{i+1} \leq_G Q_i$ und $Q_i$ stimmt auf ersten $i$ Pfaden mit $G$ überein.
	\pause
	\item I.S.: $p$ sei $i$-ter Pfad von $G$ \pause mit $p \notin Q_{i-1}$. 
		
		\pause $q$ sei Pfad aus $Q_{i-1}$ mit $q >_G p$ und kleinstem Startknoten.
		
		\pause Erhalte $Q_i$ durch Ersetzen von $q$ durch $p$ in $Q_{i-1}$.
		
		\pause Dann $Q_i \leq_G Q_{i-1}$. \pause Mit I.V.: $Q_i$ zulässig, da keine Kantenkapazität verletzt wird.
\end{itemize}
\end{frame}
	
\subsection{Identische Kapazitäten}
\begin{frame}


\end{frame}


\subsection{Willkürliche Kapazitäten}
\begin{frame}
	
\end{frame}
\end{section}