

\begin{section}{Problemdefinition}
	\begin{frame}{Problem in allgemeinen Graphen}
		\begin{definition}[Netzwerk]
			Sei $(V,E)$ ein ungerichteter Graph mit Knoten $V$ und Kanten $E$, und $c: E \to \mathbb{N}$ eine Kapazitätsfunktion.
			Das Tupel $(V,E,c)$ heißt (ungerichtetes) Netzwerk.
		\end{definition}
		\begin{definition}[\CallControl]
   			Seien $(V,E,c)$ ein ungerichtetes Netzwerk und $P$ eine (Multi-)Menge von $m \in \mathbb{N}$ Pfaden in $(V,E,c)$.

			$Q \subseteq P$ heißt {\em zulässig}, falls für alle $e \in E$ die Anzahl aller Pfade in $Q$,
			die $e$ enthalten, höchstens $c(e)$ ist.

			{\em \CallControl} besteht darin, eine zulässige Menge $Q$ maximaler Mächtigkeit zu finden.
		\end{definition}
	\end{frame}

	\begin{frame}{Call-Control in Ketten}
		\begin{definition}[Kette]
			Eine Kette $(V,E)$ ist ein Weg mit den Kanten $E=\{(v_0, v_1),\dots,(v_{n-2}, v_{n-1})\}$ mit $v_i \neq v_j$ für $i \neq j$.
		\end{definition}
	
		\only<1>{\begin{figure}[htbp]
			\centering
			\small
			\def\svgwidth{220bp}
			\input{example-chain-1.pdf_tex}
			\label{fig:k-coloring}
		\end{figure}}
		
		\only<2>{\begin{figure}[htbp]
				\centering
				\small
				\def\svgwidth{220bp}
				\input{example-chain-2.pdf_tex}
				\label{fig:k-coloring}
		\end{figure}}
	
		\only<3>{\begin{figure}[htbp]
				\centering
				\small
				\def\svgwidth{220bp}
				\input{example-chain-3.pdf_tex}
				\label{fig:k-coloring}
		\end{figure}}
	\end{frame}
\end{section}
