Nachdem wir nun optimale Lösungen für das Call-Control-Problem in Ketten und Ringen gefunden haben, modifizieren
wir in diesem Kapitel die Problemstellung ein wenig.
So geben wir nun jedem der Pfade $p$ in einer Kette oder einem Ring ein besimmtes Gewicht $\lambda_p$ und versuchen
nun, statt der Anzahl der akzeptierten Pfade, das Gesamtgewicht $\sum_{p\in Q}\lambda_p$ zu maximieren, ohne dabei
die Kapazität einer Kante zu verletzen.
Dabei verbraucht weiterhin jeder Pfad eine Kapazitätseinheit einer Kante, falls die Kante Teil des Pfades ist.

Wir werden dazu für das Problem in Ketten einen optimalen Algorithmus finden, jedoch für das Problem in Ringen
nur Approximationsalgorithmen kennenlernen. \todo{PTAS? BIPARTITE?}

\subsection{Gewichtetes Call-Control-Problem in Ketten}
Für den Fall, dass alle Kanten $e$ wieder identische Kapazitäten $c(e) = C \in \set N$ haben, stellt \todo{REFERENZ CARLISLE}
eine optimale Lösung für das äquivalente Problem des $C$-Färbens von gewichteten Intervallen vor. \todo{check that we described why we can assemble these problems}.
Dabei wird das Problem auf ein Minimum-Cost Flow Problem reduziert, und dadurch eine Laufzeit von $O(C\cdot S(m))$
erreicht, wobei $m$ die Anzahl der Intervalle (hier: Pfade) und $S(m)$ die Laufzeit eines Kürzesten-Pfade-Algorithmus
in gerichteten Graphen mit positiven Kantengewichten und $m$ Kanten ist.
Außerdem wird angenommen, dass eine sortierte Liste aller Endpunkte bereits zur Verfügung steht. \todo{Sortieren :/}

Eine Instanz der Größe $C$ des Minimum-Cost Flow Problem besteht dabei aus einem gerichteten Graphen $(V,E)$ mit
einem Startknoten $s$ und Zielknoten $t$, wobei jede Kante $e$ des Graphen die Kosten $\cost(e)$ und die Kapazität
$\capacity(e) \geq 0$ besitzt.
Eine Funktion $f\colon E \rightarrow \set R$ heißt dabei {\em Fluss}, falls  $0 \leq f(e) \leq \capacity(e)$ für alle
Kanten $e$ und in jeden Knoten $v$ (ausgenommen $s$ und $t$) genauso viel hineinfließt wie auch wieder herausfließt, d.h.
die Summe des Flusses aller eingehenden Kanten von $v$ mit der der ausgehenden Kanten übereinstimmt.
Bei $s$ beträgt der ausgehende und bei $t$ der eingehende Fluss dabei jeweils $C$.
Gesucht ist nun ein Fluss, dessen Kosten $\sum_{e\in E}f(e) \cdot \cost(e)$ minimal sind.

Übertragen auf die Nomenklatur des Call-Control-Problems, skizzieren wir im Folgenden die Reduktion unseres Problems.
Sei $v_1 < \dots < v_r$ die sortierte Liste der Endpunkte aller Pfade ohne Duplikate.
Dann erstellen wir den gerichteten Graphen mit den Endpunkten $s = v_0, \dots, v_{r+1}=t$ und erstellen
für $1 \leq i \leq r+1$ eine Kante $(v_i-1, v_i)$ mit Kapazität $C$ und Kosten $0$.
Außerdem erstellen wir für jeden Pfad $p$ eine {\em Pfadkante} $(s_p, t_p)$ mit $s_p$ Start- und $t_p$ Zielknoten von $p$,
wobei die Kosten der Kante den negativen des Pfades und die Kapazität $1$ entsprechen.
In \todo{FIGURE} ist eine solche Transformation erkennbar.
In einem Fluss $f$ repräsentieren dann die Pfadkanten $e$ mit $f(e)=1$ die zu akzeptiertenden Pfade.

Die Intuition hinter der Korrektheit dieser Problemformulierung besteht darin, den Fluss nicht als Ganzes, sondern
den Fluss der einzelnen $C$ Einheiten zu betrachten.
Dadurch erkennt man schnell, dass maximal $C$ Flusseinheiten sich überschneidende Pfade belegen können.
Für eine genauere Beschreibung und Analyse des Verfahren sowie dessen Laufzeit betrachte man \todo{referenz carlisle sec. 3}.
Wir wollen uns dagegen wieder dem Problem mit willkürlichen Kapazitäten stellen.

\begin{theorem}
    Das gewichtete Call-Control-Problem in Ketten kann in $O(N + m\cdot S(m))$ Zeitaufwand optimal gelöst werden,
    wobei $n$ die Anzahl der Knoten, $m$ die Anzahl der Kanten und $S(m)$ die Laufzeit eines Kürzesten-Pfade-Algorithmus
    in gerichteten Graphen mit positiven Kantengewichten und $m$ Kanten ist.
\end{theorem}

