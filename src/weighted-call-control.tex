Nachdem wir nun optimale Lösungen für das \CallControl-Problem in Ketten und Ringen gefunden haben, modifizieren
wir in diesem Kapitel die Problemstellung ein wenig und kümmern uns nun um das \WeightedCallControl-Problem.
So geben wir jedem der Pfade $p$ ein bestimmtes Gewicht $\omega(p) \in \set R_+$ und versuchen statt der
Anzahl der akzeptierten Pfade $Q$ das Gesamtgewicht $\omega(Q) \coloneqq \sum_{p\in Q}\omega(p)$ zu maximieren, ohne
dabei die Kapazität einer Kante zu verletzen.
Dabei verbraucht weiterhin jeder Pfad eine Kapazitätseinheit einer Kante, falls die Kante Teil des Pfades ist.

Wir werden dazu für das Problem in Ketten einen optimalen Algorithmus finden, jedoch für das Problem in Ringen
nur einen Approximationsalgorithmus von Güte 2 kennenlernen.

\subsection{\WeightedCallControl-Problem in Ketten}\label{subsec:weighted-call-control-in-chains}
Für den Fall, dass alle Kanten $e$ wieder identische Kapazitäten $c(e) = C \in \set N$ haben, stellen Carlisle und Lloyd
in~\cite{carlisle} eine optimale Lösung für das äquivalente Problem des $C$-Färbens von gewichteten Intervallen vor.
Dabei wird das Problem auf ein Minimum-Cost Flow Problem reduziert, und dadurch eine Laufzeit von $\mathcal O(C\cdot S(m))$
erreicht, wobei $m$ die Anzahl der Intervalle (hier: Pfade) und $S(m)$ die Laufzeit eines Kürzesten-Pfade-Algorithmus
in gerichteten Graphen mit positiven Kantengewichten und $m$ Kanten ist.
Außerdem wird wieder angenommen, dass bereits eine sortierte Liste aller Endpunkte zur Verfügung steht.

Eine Instanz der Größe $C$ des Minimum-Cost Flow Problem besteht dabei aus einem gerichteten Graphen $(V,E)$ mit zwei
ausgezeichneten Knoten $s$ und $t$, wobei jede Kante $e$ des Graphen die Kosten $\cost(e)$ und die Kapazität
$\capacity(e) \geq 0$ besitzt.
Eine Funktion $f\colon E \rightarrow \set R$ heißt {\em Fluss}, falls $0 \leq f(e) \leq \capacity(e)$ für alle
Kanten $e$ und in jeden Knoten $v$ (ausgenommen $s$ und $t$) genauso viel hineinfließt wie auch wieder herausfließt, d.h.
die Summe des Flusses aller eingehenden Kanten von $v$ mit der der ausgehenden Kanten übereinstimmt.
Bei $s$ beträgt der ausgehende und bei $t$ der eingehende Fluss dabei jeweils $C$.
Gesucht ist nun ein Fluss, dessen Kosten $\sum_{e\in E}f(e) \cdot \cost(e)$ minimal sind.

Übertragen auf das \WeightedCallControl-Problem, skizzieren wir für das Kettennetzwerk $(V,E,c)$
und der Menge $P$ der gegebenen Pfade in $(V,E)$ im Folgenden die Reduktion unseres Problems.
Die Knoten der Kette $(V,E)$ seien wieder wie gewohnt von links nach rechts durchnummeriert.
Sei $v_0 < \dots < v_r$ die sortierte Liste der Endpunkte aller Pfade ohne Duplikate.
Wir bauen den gerichteten Graphen als gerichteten Pfad mit den Knoten $\{s = v_0, \dots, v_r=t\}$ und erstellen
für $1 \leq i \leq r$ eine Kante $(v_{i-1}, v_i)$ mit Kapazität $C$ und Kosten $0$.
Zusätzlich erstellen wir für jeden Pfad $p \in P$ eine sogenannte {\em Pfadkante} $(s_p, t_p)$ mit $s_p$ Anfangs- und $t_p$ Zielknoten von $p$,
wobei die Kosten der Kante dem negativen Gewicht des Pfades und die Kapazität der Kante $1$ entsprechen.
%In \todo{FIGURE} ist eine solche Transformation erkennbar.
In einem Fluss $f$ repräsentieren dann die Pfadkanten $e$ mit $f(e)=1$ die akzeptierten Pfade.

Die Intuition hinter der Korrektheit dieser Problemformulierung besteht darin, den Fluss nicht als Ganzes, sondern
den Fluss der einzelnen $C$ Einheiten zu betrachten.
Dadurch erkennt man schnell, dass maximal $C$ Flusseinheiten sich überschneidende Pfade belegen können.
Für eine genauere Beschreibung und Analyse des Verfahren sowie dessen Laufzeit betrachte man~\cite{carlisle}.
Wir wollen uns dagegen dem Problem mit willkürlichen Kapazitäten widmen.

\begin{theorem}
    Das gewichtete Call-Control-Problem einer Kette $(V, E)$ mit willkürlichen Kapazitäten kann in $\mathcal O(n + m\cdot S(m))$
    Zeitaufwand optimal gelöst werden.
    Dabei ist $n = |V|$, $m=|E|$ und $S(m)$ die Laufzeit eines Kürzesten-Pfade-Algorithmus in gerichteten Graphen mit
    positiven Kantengewichten und $m$ Kanten.
\end{theorem}
\begin{proof}
    Wir führen auch dieses Problem wieder auf ein Problem mit identischen Kapazitäten zurück.
    Dabei ist $C = \max_{e \in E}c(e)$ die maximale Kantenkapazität.
    Wie in Abschnitt~\ref{subsec:anpassenAnWillkürlicheKapazitäten} müssen die zusätzlichen $C - c(e)$ Kapazitäten
    aller Kanten $e$ durch Platzhalterpfade aufgefüllt werden.
    Erhalten wir eine optimale Lösung, in der alle Platzhalterpfade akzeptiert wurden, so ist die Menge der akzeptierten
    Originalpfade wieder eine optimale Lösung für das Ursprungsproblem.

    Um dafür zu sorgen, dass alle Platzhalterpfade $p$ in jedem Fall akzeptiert werden, setzen wir
    $\omega(p) \coloneqq G + 1$, wobei $G$ die Summe der Gewichte aller Originalpfade ist.
    Nun hat der Algorithmus für identische Kapazitäten keine andere Wahl mehr, als jeden Platzhalterpfad zu akzeptieren,
    da jede Menge an Originalpfaden, die er stattdessen akzeptieren könnte, ein geringeres Gewicht hat und es nach
    Konstruktion genügend Kapazitäten gibt, um jeden Platzhalterpfad zu akzeptieren.

    Für die Sortierung der Endpunkte aller Pfade benötigen wir noch $\mathcal O(n+m)$ Zeit und erhalten insgesamt eine Laufzeit
    von $\mathcal O(n+m\cdot S(m))$.
\end{proof}

\subsection{Approximationsalgorithmus von Güte 2 für Ringe}
In diesem Abschnitt betrachten wir einen simplen Approximationsalgorithmus von Güte 2 für das Gewichtete
Call-Control-Problem in Ringen, d.h.\ einen Algorithmus der die optimale Lösung bis auf den Faktor 2 annähert.
Dabei gehen wir wie folgt vor:

Wir suchen die Kante $e$ des Rings mit der geringsten Kapazität $c_{\min}$ und betrachten die beiden Pfadmengen
$Q_{\overline e}^*$ und $Q_{e}^*$.
Wir berechnen $Q_{\overline e}^*$ mit Hilfe von Abschnitt~\ref{subsec:weighted-call-control-in-chains} als die
optimale Lösung des Gewichteten Call-Control-Problem der Kette, die durch
Weglassen der Kante $e$ entsteht, mit allen Pfaden, die $e$ nicht enthalten.
Dagegen stellt $Q_e^*$ die Menge der $c_{\min}$ Pfade größten Gewichts, die $e$ enthalten, dar.
Zwischen $Q_{\overline e}^*$ und $Q_{e}^*$ wird nun die Menge mit größerem Gesamtgewicht als
Approximationslösung von Güte 2 gewählt.

\begin{theorem}
    Der beschriebene Algorithmus birgt eine Approximation von Güte 2 für das Gewichtete Call-Controll-Problem in Ringen.
\end{theorem}
\begin{proof}
    Es beschreibe $e$ die Kante aus dem Algorithmus mit der geringsten Kapazität.
    Zunächst sehen wir, dass $Q_{\overline e}^*$ und $Q_e^*$ geeignete Mengen darstellen.
    Das ist klar, da $Q_{\overline e}^*$ sogar eine geeignete Menge für die Kette ohne $e$ ist und $Q_e^*$ nur $c_{\min}$
    Pfade enthält, weshalb in beiden Fällen keine Kapazitäten verletzt werden.

    Sei $\OPT$ eine optimale Lösung für \WeightedCallControl\ im Ring.
    $\OPT$ ist die disjunkte Vereinigung der beiden Mengen $\OPT_e \coloneqq \{p \in \OPT \mid p\ \text{enthält}\ e\}$ und
    $\OPT_{\overline e} \coloneqq \{p\in\OPT\mid p\ \text{enthält}\ e\ \text{nicht}\}$.
    Es gilt insbesondere, dass $\omega(Q_{\overline e}^*) \geq \omega(\OPT_{\overline e})$ und
    $\omega(Q_e^*) \geq \omega(\OPT_{e})$, da $Q_{\overline e}$ und $Q_e$ in ihren Bereichen maximal gewählt wurden.
    Daher folgt für die berechnete Lösung $Q^*$:
    \[2 \cdot \omega(Q^*) = 2 \cdot  \max\{\omega(Q_e^*), \omega(Q_{\overline e}^*)\} \geq
    \omega(\OPT_e) + \omega(\OPT_{\overline e}) = \omega(\OPT) \]
\end{proof}
