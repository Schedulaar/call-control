Die vier Autoren des Artikels \glqq Call Control in Rings\grqq\ haben es geschafft, das \CallControl-Problem in Ketten mit
einem Algorithmus in linearer Zeit zu lösen.
Diesen Algorithmus konnte man dann beim Problem in Ringen elegant einsetzen, um so zu einem Algorithmus zu gelangen, der mit polynomiellem Zeitaufwand
eine optimale Lösung des \CallControl-Problems in Ringen berechnet.
Dies ist vor allem aus diesem Grund interessant, da ein ähnlich wirkendes Problem, bei dem es darum geht den größtmöglichen $k$-färbbaren 
Teilgraphen in Kreisbogengraphen zu berechnen, {\em NP}-hart ist, wie in~\cite{circular-arc} nachgewiesen wurde.
Ein Kreisbogengraph $(V,E)$ mit Knotenmenge $V$ ist dabei wie folgt definiert: 
Existiert eine Familie von Kreisbögen $(K_x)_{x\in V}$ um einen gemeinsamen Punkt mit festem Radius 
und gilt $(x,y)\in E \Leftrightarrow K_x \cap K_y \neq \emptyset$, so ist $(V,E)$ ein Kreisbogengraph.

Für das \WeightedCallControl-Problem wurde anhand der Vorarbeit in~\cite{carlisle} ein effizienter Algorithmus in Ketten sowie ein darauf
aufbauender Approximationsalgorithmus von Güte 2 für das Problem in Ringnetzwerken entwickelt.
Außerdem zeigen die Autoren in~\cite{paper}, dass eine beliebig nahe Approximation in polynomieller Zeit gefunden werden kann.
Dazu wird ein sogenanntes Polynomial-time approximation scheme (kurz PTAS) entwickelt, bei dem ein Parameter $\epsilon \in \set R_+$ gegeben ist
und der Algorithmus eine Lösung berechnet, dessen Gesamtgewicht um einen Faktor von $1-e$ von dem der optimalen Lösung abweicht.
Außerdem ist das \WeightedCallControl-Problem in Ringen sogar mindestens so schwer wie das exact-matching-Problem in einem bipartiten Graphen (siehe \cite{hochbaum-levin}),
also einem Graphen, der in zwei unabhängige disjunkte Teilgraphen U und V geteilt werden kann, sodass jeder Knoten in U mit mindestens einem Knoten in V und andersherum verbunden ist.
Von diesem ist jedoch immer noch unbekannt, ob es in $P$ liegt oder ob es ein {\em NP}-hartes Problem darstellt.